\begin{resume}

  \section*{发表的学术论文} % 发表的和录用的合在一起

  \begin{enumerate}[{[}1{]}]
  \addtolength{\itemsep}{-.36\baselineskip}%缩小条目之间的间距,下面类似
  \item \textbf{Qiang Lan}, Zelong Wang, Mei Wen, Chunyuan Zhang, Yijie Wang.
  High performance Implementation of 3D Convolutional Neural Networks on a GPU.
  Journal of Computational Intelligence and Neuroscience, 2017.
  (SCI)
    \item \textbf{Qiang Lan}, Linbo Qiao, Yijie Wang.
  Stochastic extra-gradient based alternating direction methods for graph-guided regularized minimizations.
  Frontiers of Information Technology \& Electronic Engineering , 2017.
  (SCI)
  \item \textbf{Qiang Lan}, Changqing Xun, Mei Wen, Huayou Su, Lifang Liu, Chunyuan Zhang.
  Improving Performance of GPU Specific OpenCL Program on CPUs.
  The Thirteenth International Conference on Parallel and Distributed Computing Applications and Technologies (PDCAT 2012), 2013.
  (EI, CCF C类会议)
    \item \textbf{Qiang Lan}, Namit Gaur, Johannes Langguth, Xing Cai.
  Towards Detailed Tissue-Scale 3D Simulations of Electrical Activity and Calcium Handling  in the Human Cardiac Ventricle.
  The 15th International Conference on Algorithms and Architectures for Parallel Processing (ICA3PP 2015), 2015.
  (EI , CCF C类会议)
      \item Johannes Langguth, \textbf{Qiang Lan}, Namit Gaur, Xing Cai.
  Accelerating Detailed Tissue-scale 3D Cardiac Simulations using Heterogeneous CPU-Xeon Phi Computing.
  International Journal of Parallel Programming, 2016.
  (SCI)
   \item  Johannes Langguth, \textbf{Qiang Lan}, Namit Gaur, Xing Cai, Mei Wen, and Chun-yuan Zhang.
  Enabling Tissue-scale Cardiac Simulations using Heterogeneous Computing on Tianhe-2.
  The 22nd IEEE International Conference on Parallel and Distributed Systems ( ICPDS), 2016.
  (EI, CCF C类会议)
  \item Zelong Wang, \textbf{Qiang Lan}, Dafei Huang,Mei Wen.
  Combining FFT and Spectral-Pooling for Efficient Conrolution Neural Network Model.
 The 2nd International Conference on Artificial Intelligence and Industrial Engineering( ICAII), 2016.
  (EI)
 \item Yang Shi, \textbf{Qiang Lan}, Hao Fang, Mei Wen.
  Accelerating CNN’s Forward Process on Mobile GPU Using OpenCL.  ICDIP, 2016.
  (EI )
    \end{enumerate}
    
    
%    \begin{enumerate}[{[}1{]}]
%  \addtolength{\itemsep}{-.36\baselineskip}%缩小条目之间的间距,下面类似
%  \item 第一作者.
%  High performance Implementation of 3D Convolutional Neural Networks on a GPU.
%  Journal of Computational Intelligence and Neuroscience, 2017.
%  (SCI)
%    \item 第一作者.
%  Stochastic extra-gradient based alternating direction methods for graph-guided regularized minimizations.
%  Frontiers of Information Technology \& Electronic Engineering , 2017.
%  (SCI)
%  \item 第一作者.
%  Improving Performance of GPU Specific OpenCL Program on CPUs.
%  The Thirteenth International Conference on Parallel and Distributed Computing Applications and Technologies (PDCAT 2012), 2013.
%  (EI, CCF C类会议)
%    \item 第一作者.
%  Towards Detailed Tissue-Scale 3D Simulations of Electrical Activity and Calcium Handling  in the Human Cardiac Ventricle.
%  The 15th International Conference on Algorithms and Architectures for Parallel Processing (ICA3PP 2015), 2015.
%  (EI , CCF C类会议)
%     \item 第二作者.
%  Accelerating Detailed Tissue-scale 3D Cardiac Simulations using Heterogeneous CPU-Xeon Phi Computing.
%  International Journal of Parallel Programming, 2016.
%  (SCI)
%   \item  第二作者.
%  Enabling Tissue-scale Cardiac Simulations using Heterogeneous Computing on Tianhe-2.
%  The 22nd IEEE International Conference on Parallel and Distributed Systems ( ICPDS), 2016.
%  (EI, CCF C类会议)
%  \item 第二作者.
%  Combining FFT and Spectral-Pooling for Efficient Conrolution Neural Network Model.
% The 2nd International Conference on Artificial Intelligence and Industrial Engineering( ICAII), 2016.
%  (EI)
% \item 第二作者.
%  Accelerating CNN’s Forward Process on Mobile GPU Using OpenCL.  ICDIP, 2016.
%  (EI )
%    \end{enumerate}
    
    
    
\end{resume}
