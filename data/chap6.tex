%*********************第六章******************
\chapter{总结与展望}
\section{工作总结}
视觉跟踪是计算机视觉领域的核心方向,是大量人工智能高层应用的基石。
视觉跟踪的研究在近三十年里长盛不衰,不仅因为其具有不可替代的理论意义和实用价值,
而且源于它所面临的持续不断的挑战,如各种复杂的干扰因素、物体的随机运动和形变、对速度要求的不断提高等。
为了应对这些挑战,本文以高性能视觉跟踪关键技术作为研究课题,
希望从两个方面来实现高性能的视觉跟踪\pozhehao 视觉跟踪的高性能算法和跟踪算法的高性能实现。

在视觉跟踪的高性能算法方面,本文主要针对当前跟踪算法对物体大小变化适应力不足的问题,
提出了将``物体类别无关''的目标候选生成器和视觉跟踪算法将结合的方法。
在跟踪算法部分,本文使用了高效且准确的基于相关滤波的跟踪器,并对其进行了特征整合和鲁棒模型更新等优化;
对于目标候选生成器,本文进行了参数调优,并用配备了一个目标候选过滤步骤,
以生成数量较少但足够灵活的目标候选供相关滤波器辨别。
最终得到的跟踪器展现出了很强的鲁棒性和适应力,同时还达到了令人满意的跟踪速度。

为了进一步研究目标候选这一检测领域的常用方法在视觉跟踪中的作用,
本文还将多个现有的目标候选生成器面向跟踪任务进行了适配,并将它们合理地嵌入了跟踪器中。
此外,针对当前高质量目标候选生成器在跟踪任务中的弱点,
本文还对其进行了额外的背景抑制优化。
通过大量的实验,本文证明了目标候选的质量和跟踪精度间存在着正相关的关系,
同时也确认了对于目标候选生成器的背景抑制优化是十分有效的。

在视觉跟踪算法的高性能实现方面,
本文以TLD跟踪算法为例,以OpenCL作为编程模型,阐述了视觉跟踪算法
在异构平台下的高性能实现所需的关键技术和面临的问题挑战。
本文并行化了TLD算法的计算密集部分和瓶颈部分,包括Fern随机森林的特征提取和分类过程,
最邻近分类的NCC计算过程,以及学习过程中的重叠率计算和正负样本提取。
此外,本文还将Fern随机森林的处理和LK光流跟踪在不同计算设备上进行了重叠执行。
最终的高性能实现取得了令人满意的整体加速比,完全满足实时处理的需求,
但同时也发现了数据传输开销较大、OpenCL Kernel程序的性能移植性不佳的问题。

针对OpenCL Kernel程序的性能移植性问题,本文提出了一套新的代码转换方法。
该方法专注于GPU到多核/众核CPU的性能移植性,
能够大幅提升为GPU专门编写的Kernel程序在CPU上的执行性能。
该方法借助于本文新提出的数组访问描述式,在工作项折叠过程中,
能够消除所有的冗余局部存储数组和对应的同步。
后继优化过程中,该方法不仅从原GPU特定Kernel中提取并行性和局部性信息,
还会考虑目标CPU的体系结构细节,以进一步提升Kernel程序在CPU 上的性能。
实验显示,对于GPU 特定的Kernel程序,若使用包含本文方法的新OpenCL运行时,
将获得超越Intel官方运行时的性能。

通过在视觉跟踪高性能算法方面的研究,
本文获得了一个将目标候选方法和跟踪算法进行结合的通用方法,并揭示了目标候选质量对跟踪精度的作用规律。
这些成果将指导物体检测技术在视觉跟踪中的进一步应用。
通过在跟踪算法高性能实现方面的工作,
本文获得了一个基于OpenCL的TLD跟踪算法完整高性能实现。
由于TLD实质上是一个跟踪框架,并且其各个模块可以自由替换为具有类似功能的算法,
本文的高性能实现可以作为大量高性能跟踪器的实现基础。
此外,借助于本文OpenCL Kernel程序的性能移植性提升方法,跟踪算法高性能实现的移植性也将获得提升,
在异构平台上实现高性能跟踪器将更为容易。

\section{未来研究方向}
尽管本文取得了一些研究成果,但是仍然存在着不完善之处,需要进一步的解决。
此外,一些新的研究点也值得深入探索。
未来的研究计划有以下几点:
\begin{compactitem}
\item[1.]
本文在将目标候选和跟踪算法进行结合时,所使用的方法是通用的。
但是本文为了研究目标候选对于视觉跟踪的作用,仅采用了一种跟踪算法。
未来将把本文优化后的EdgeBoxes和不同的跟踪算法相结合,以进一步探索目标候选对于视觉跟踪的适用性规律。

\item[2.]
本文中,即使进行了面向跟踪任务的优化,目标候选方法仍然仅作为密集采样运动模型的一个补充。
未来计划对目标候选方法进行全面修改,或者提出专门用于视觉跟踪的目标候选生成方法,
将目标候选生成器从生成``物体检测目标候选'',转变为生成``物体跟踪目标候选'',
从而形成一类全新的运动模型。

\item[3.]	
本文中,基于OpenCL的高性能TLD算法实现还远没有达到最佳性能。
未来将进行进一步的优化,包括:对TLD代码进行重构,调整中间数据的存储方式,减少数据重组的时间开销;
通过在进行计算的同时传输Kernel所需的数据,隐藏数据传输的时间开销;
进行更加精细的并行优化,如考虑最新的GPU架构特点,优化PTX层次的代码等。

\item[4.]	
本文提升GPU特定Kernel程序在CPU上的性能移植性的相关方法还不够完善,
未来计划进行如下工作:
将工作组间的静态调度方法转换为动态的负载均衡调度;
考虑同一个CPU核上的工作组间并行化,进一步提升CPU的利用率;
通过加入编译指导语句的方式,支持对间接访存的访存分析。

\item[5.]	
将本文的研究成果进行整合,
实现一个高性能、可移植的视觉跟踪编程框架。
该框架提供算法层的编程接口,用户可
以根据跟踪应用的具体需求,选择不同的模块作为运动模型、观察模型等,并且可以对各
个模块的参数进行设置;
该框架还提供实现层的编程接口,用户可以将不同模块的计算负
载指派到异构平台中不同的设备上,并且计算负载在不同设备上均能高效执行。
\end{compactitem}

