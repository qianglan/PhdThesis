%*********************第六章******************
\chapter{总结与展望}
\section{工作总结}
本课题专注于异构计算的研究,无论是在高性能计算领域还是通用计算领域,异构计算都变得越来越重要了。本课题选取的两个应用分别是心脏模拟和深度学习领域比较热门的卷积神经网络应用。下面对这两个应用特点进行概述。

首先是心脏组织模拟应用的特点,归纳如下:

\begin{compactitem}
\item[1.]
由于本课题采用的是非常精细的细胞模型,心脏细胞内的计算量非常大,因此属于典型的计算密集型应用。虽然也有部分计算,比如细胞内dyad间电压的扩散过程,这是典型的stencil计算,这部分计算属于访存密集型的,但这部分计算只占细胞内所有计算时间的一小部分。因此,主要的计算类型是密集型计算。

\item[2.]
部分计算是控制语句主导的。

\item[3.]
心脏组织的模拟具有多层并行性。

\item[4.]
心脏组织模拟如果需要模拟更加复杂的行为时,负载会出现不均衡的问题。

\item[5.]
心脏组织模拟通信开销小。只需要传输电压值

\item[6.]
精度敏感型。

\end{compactitem}

属于不同领域的应用,心脏模拟是科学计算领域,计算对精度要求比较高,采用的都是双精度数据类型,是典型的计算密集型应用。

卷积神经网络中的卷积计算虽然通过算法的改进可以降低浮点计算量,但核心计算部分仍然是矩阵乘运算,适合在异构平台上进行加速实现,本课题的改进算法中除了矩阵乘运算,还有部分计算主要涉及访存操作,是典型的访存密集型。

本课题针对这两种应用分别在两种不同的异构平台上进行映射实现。

\section{未来研究方向}



