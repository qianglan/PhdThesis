%*********************第六章******************
\chapter{总结与展望}
\section{工作总结}
本课题专注于异构计算的研究,无论是在高性能计算领域还是通用计算领域,异构计算都变得越来越重要了。本课题选取的两个应用分别是心脏模拟和深度学习领域比较热门的卷积神经网络应用。下面对这两个应用特点进行概述。

首先是心脏组织模拟应用的特点,归纳如下:

\begin{compactitem}
\item[1.]
主体计算是计算密集型。由于本课题采用的是非常精细的细胞模型,心脏细胞内结构复杂, 含有大量的dyad单元,每个dyad单元内又要模拟100多个通道,因此,心脏细胞内的计算量巨大,属于典型的计算密集型应用。虽然也有部分计算,比如细胞内dyad间电压的扩散过程,这是典型的stencil计算,这部分计算属于访存密集型的,但这部分计算只占细胞内所有计算时间的一小部分。因此,主要的计算类型是密集型计算。

\item[2.]
部分计算是控制语句主导的。对于细胞内的dyad单元的通道状态的模拟,按照直接的方法就是对每个dyad单元单独地进行模拟,每一个dyad单元中的通道状态模拟都涉及到中间计算结果与随机数的比较,比较控制语句在某些体系结构中是非常不高效的,比如在MIC加速器上,比较控制语句性能久很差。本课题采用的是二项分布随机采样模拟,这是一个随机采样的过程,对每个dyad的所有通道看作一个整体考虑,大大减少了控制比较语句的执行次数。不过还是无法完全避免所有的控制比较语句,所以这部分计算还是占了一定的开销。

\item[3.]
心脏组织的模拟具有多层并行性。心脏组织模拟应用表现出非常丰富的并行性,对计算的需求非常巨大,需要大规模并行系统才能满足。而心脏组织模拟的计算特点非常适合映射到大规模并行计算系统中,首先心脏组织被划分成规则的网格,每一个网格内的细胞计算由一个计算节点负责;其次,针对有多个加速器构成的异构节点,可以根据异构节点中主机CPU和加速器的计算性能比按比例将网格中的细胞进行划分;对于无论是多核CPU还是众核CPU负责的细胞计算,又可以采用OpenMP对细胞进行并行计算;最后每个小核在对单个细胞进行计算时,可以利用单核的SIMD特性对细胞内dyad单元进行向量化。

\item[4.]
心脏组织模拟如果需要模拟更加复杂的行为时,负载会出现不均衡的问题。如果需要模拟复杂的行为,需要给心脏组织的不同部位进行刺激,有时可能需要在不同的时刻进行刺激。由于刺激的细胞将电压传导到周围的细胞,那些已经被传导的细胞的行为与未被传导的细胞的行为是不同的,在模拟它们时所需计算量是有差异的。模拟的行为越复杂,心脏组织不同部位的模拟所需计算量的差异就越大,负载将变得越来越不均衡。本课题只模拟了简单的行为,因此,负载均衡问题不是特别突出。

\item[5.]
心脏组织模拟通信开销小。在心脏组织模拟应用中,发生在每个细胞内的计算相互独立,细胞间唯一发生联系的是细胞内的电压值,细胞内电压的计算与相邻细胞的电压值相关,因此,每个计算节点负责的网格细胞中处于外围的细胞的电压值都需要与处于相邻的网格外围细胞的电压值进行交换,每次时间迭代各个细胞的电压值都需要更新,因此,每个时间迭代步都涉及到电压值的通信。但相比每次迭代发生计算量,通信开销所占的比例是相当少的,这也是本课题在大规模节点中能取得很好扩展性的一个重要原因。

\item[6.]
精度敏感型。心脏组织模拟中采用精细的细胞模型,对精度要求极高,其中的变量都采用双精度浮点表示。高精度浮点计算将占用更多的计算资源,需要使用更宽的寄存器,意味着SIMD向量计算单元每次能处理的元素将变少。

\end{compactitem}

深度学习利用大量的训练数据对神经网络进行训练,在某些应用领域表现出了非常好的效果。现在比较流行的神经网络是卷积神经网络,目前各种卷积神经网络的设计表明,网络层次越深,精度越高,但问题就是带来计算量的增加。卷积神经网络涉及的计算与心脏组织模拟中的计算在某些方面具有相似的特点,比如都是计算密集型计算。但也表现出不一样的特点,具体如下:
\begin{compactitem}
\item[1.]核心计算其实就是矩阵乘运算。

\item[2.]前向计算过程对延迟要求低。

\item[3.]前向计算过程具有多种并行模式,不同并行模式涉及不同的通信模式。

\item[4.]大规模训练具有大规模计算节点的需求。

\item[5.]对数值表示的精度要求相对较低。
\end{compactitem}

卷积神经网络中的卷积计算虽然通过算法的改进可以降低浮点计算量,但核心计算部分仍然是矩阵乘运算,适合在异构平台上进行加速实现,本课题的改进算法中除了矩阵乘运算,还有部分计算主要涉及访存操作,是典型的访存密集型。

本课题针对这两种应用分别在两种不同的异构平台上进行映射实现。

\section{未来研究方向}



