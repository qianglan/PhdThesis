%*********************第二章******************
\chapter{基于大规模多核CPU系统的心脏组织的3D精细模拟}
\label{ICA3PP}

\section{引言}
细胞内钙离子处理功能异常被认为是几个心脏病理(比如心脏衰竭\upcite{marks2013calcium,kubalova2005abnormal},心脏肥大\upcite{berridge2006remodelling},心肌病\upcite{pieske1995alterations}以及肌浆网内钙离子释放处理紊乱\upcite{liu2006arrhythmogenesis,priori2011inherited})的极有可能的诱因。上述心脏疾病中的大多都是源于亚细胞内微米级和纳米级上钙离子释放处理功能不正常造成的,表现为由t-型管畸形\upcite{louch2004reduced,louch2006t,van2011disrupted}和单兰尼碱受体功能紊乱\upcite{liu2006arrhythmogenesis,jiang2005enhanced}造成心脏的病理。

在过去的几年里,数值方法和计算技术的进步使得心脏细胞的电生理学和钙离子处理模型得到了快速发展,这些模型也开始考虑亚细胞中的随机钙离子释放过程\upcite{restrepo2008calsequestrin,gaur2011multiscale,nivala2012computational,williams2011dynamics}的离散特性。新一代的钙离子处理和动作电位模型在研究心律失常的机制和预防具有非常重要的价值,心律失常主要因为细胞内纳米级钙离子释放通道和dyadic单元功能的紊乱进而影响亚细胞级的膜电位的不正常而造成的,膜电位的异常通常表现为去极化延迟、过早地去极化\upcite{song2015calcium}以及心脏电交替变化\upcite{restrepo2008calsequestrin,nivala2012calcium,nivala2015t}。虽然这些发展能让我们从不同尺度的动作电位对心律失常进行理解,这些尺度可以从单通道到整个细胞,然而我们仍然面临很多挑战。

首先,心律失常发生在心脏组织级和器官级。从细胞这一层级的研究对心脏组织和器官的心律失常进行理解推断往往不是特别清楚,有时反而不利于进行研究。其次,从心脏组织这个层次进行研究已经被证明对计算有极大的需求。一个典型的人类心脏大概有$2 \times 10^9 $个细胞\upcite{adler1974cell},每个细胞内大概有$10^6$个RyRs和约$10^5$个L型通道,这些通道的功能就是对膜电位和细胞内各个单元里的钙离子浓度进行随机的作出响应\upcite{cheng1993calcium}。在心脏组织级这个层次的真实模拟不仅需要大量的并行计算节点,也需要复杂的算法设计,才能在合理的时间内对心脏的生理过程进行模拟。并且,到目前为止,还没有在心脏组织级这个层次采用精细的钙离子处理细胞模型对心律失常的机制进行研究。

本章主要介绍了在心脏组织中的电活动和钙离子处理的3D并行模拟器,所谓的并行模拟器就是对模拟过程在大规模的多核CPU系统中进行并行加速。本章首先会介绍关于心脏组织模拟的背景以及相关工作;然后将具体介绍本课题研究中所采用的数学模型,包括心脏组织级的数学模型和细胞内的数学模型;第三部分将介绍心脏组织模拟在多核CPU集群上的实现过程,包括节点内和节点间的并行实现以及单个细胞内的模拟过程的实现;最后是对心脏组织内各个单元的电位变化以及钙离子浓度变化的模拟,实验结果表明,本课题提出的精细的细胞模型能够很好地对心脏组织内的活动进行模拟。

%\section{相关研究}
%\subsection{心脏组织模拟的数学模型}
%\subsection{心脏组织模拟的并行实现}


\section{心脏组织3D模拟的数学模型和数值方法}

\subsection{组织级的数学模型}
心脏组织级模拟是通过公式 \ref{tissuevoltage}来描述的,该模型也叫单域模型。

\begin{equation}
\frac{\partial V_{m}}{\partial t}=\frac{-I_\mathrm{ion}}{C_{m}} + D_{x}\frac{\partial^{2}V_{m}}{\partial x^{2}}+D_{y}\frac{\partial^{2}V_{m}}{\partial  y^{2}}+D_{z}\frac{\partial^{2}V_{m}}{\partial z^{2}},
\label{tissuevoltage}
\end{equation}

公式\ref{tissuevoltage}中$V_{m}$为膜电位,$I_\mathrm{ion}$为组织模拟中主要涉及的离子流动产生的电流,$C_{m}=1\mu F cm^{-2}$ 是细胞内膜间电容,$D_{x}=D_y=D_z=0.2 $ $mm^{2}/ms$是电压在三维空间上各个坐标方向的扩散系数。心脏3D组织级模型中,3D组织是由众多心脏细胞构成的,而心脏细胞的模型是采用\ref{cellmodel}中描述的模型。

有限差分法结合算子分裂方法\upcite{qu1999advanced}被用来对公式\ref{tissuevoltage}进行离散化。这意味着扩散项与离子的电流项$I_\mathrm{ion}$分别进行处理。电流项需要通过求解每个细胞的精细模型方程而得到。细胞内的精细模型将在\ref{cellmodel}进行介绍。

\subsection{细胞级的数学模型}
\label{cellmodel}

 本课题借鉴了文献\upcite{gaur2011multiscale}在心肌细胞内采用随机钙离子处理的多尺度模型。为了模拟人类心脏组织细胞,本课题使用O'Hara-Rudy (ORd)的心脏电位公式\upcite{o2011simulation}替换了文献\upcite{gaur2011multiscale}中的电生理电流公式,ORd模型是针对健康人类的心脏细胞的模拟,而文献\upcite{gaur2011multiscale}中描述的是豚鼠的模型。
 
 \begin{figure}[ht]
\center
\includegraphics[scale=0.5]{figs/state.pdf}
\caption{The eight possible transitions between four states of a RyR, where the labels on the arrows indicate the
probabilities of the transitions.}
\label{ryrstates}
\end{figure}

在每个细胞里,有约$10000$个钙离子释放单元也叫dyads,这些钙离子释放单元按照$100\times 10\times 10$的三维网格进行排列。每个dyad包含五个不同的小单元,钙离子会在这些小单元里停留,详细的公式和参数可以参考文献\upcite{gaur2011multiscale}。每个dyad空间里,包含15个L-型钙离子通道和$100$个RyRs,它们是随机地活动的。在任何时刻,每个RyR会处于四个状态中的一个,这四个状态用C1、~C2、~C3以及~O1表示。图\ref{ryrstates}中给出了各个状态间可能的转换关系,以及状态间转换的概率,图\ref{ryrstates}状态转换的概率参数是与每个RyR内的钙离子浓度相关的,除了其中从~O1到~C1以及从~C3到~C2的转换概率为常数。处于打开状态(即处于~O1状态)的RyR的个数直接影响到钙离子的流量,进而影响每个细胞内的电压。

dyad中的钙离子浓度可以通过公式\ref{eq:1}至公式\ref{eq:10}来求解。

{
\allowdisplaybreaks
 \begin{eqnarray}
  ca_{\mathrm{ds}}& = & (J_{\mathrm{rel}} + J_{\mathrm{lca}} + ca_{\mathrm{ss}} / \tau_{\mathrm{efflux}})\times \tau_{\mathrm{efflux}}\label{eq:1}\\
  \frac{d\, ca_{\mathrm{ss}}}{dt}& = & \overline{B_{\mathrm{ss}}}\left(  J_{\mathrm{NCX}} + J_{\mathrm{diff\_myo\_ss}} + J_{\mathrm{diff\_ds\_ss}}\right) \label{eq:2}\\
  \frac{d\,ca_{\mathrm{JSR}}}{dt}& = & \overline{B_{\mathrm{JSR}}}\left(J_{\mathrm{rel}} + J_{\mathrm{diff\_NSR\_JSR}}\right) \label{eq:3}\\
  \frac{d\,ca_{\mathrm{NSR}}}{dt}& = & J_{\mathrm{up}} - J_{\mathrm{leak}} - J_{\mathrm{diff\_NSR\_JSR}}  \label{eq:4}\\ 
  \frac{d\,ca_{\mathrm{myo}}}{dt}& = & \overline{B_{\mathrm{myo}}}\left(J_{\mathrm{cab}} + J_{\mathrm{pca}} + J_{\mathrm{NCX}} - J_{\mathrm{up}} + J_{\mathrm{leak}} - J_{\mathrm{diff\_myo\_ss}}\right) \label{eq:5}\\
  Ca_{\mathrm{ds}}& = & (J_{\mathrm{rel}} + J_{\mathrm{lca}} + Ca_{\mathrm{ss}} / \tau_{\mathrm{efflux}})\times \tau_{\mathrm{efflux}}\label{eq:6}\\  
  \frac{\partial Ca_{\mathrm{ss}}}{\partial t}& = & \overline{B_{\mathrm{ss}}} \left(J_{\mathrm{NCX}} + J_{\mathrm{diff\_myo\_ss}} + J_{\mathrm{diff\_ds\_ss}}
+ D_{Ca}\nabla^2Ca_{\mathrm{ss}}\right) \label{eq:7}\\
  \frac{d\,Ca_{\mathrm{JSR}}}{dt}& = & \overline{B_{\mathrm{JSR}}}\left(J_{\mathrm{rel}} + J_{\mathrm{diff\_NSR\_JSR}}\right) \label{eq:8}\\
  \frac{\partial Ca_{\mathrm{NSR}}}{\partial t}& = & J_{\mathrm{up}} - J_{\mathrm{leak}} - J_{\mathrm{diff\_NSR\_JSR}}  +  D_{\mathrm{SR}}\nabla^2 Ca_{\mathrm{NSR}}\label{eq:9}\\					
  \frac{\partial Ca_{\mathrm{myo}}}{\partial t}& = &
                                                     \overline{B_\mathrm{myo}}\left(J_\mathrm{cab} + J_\mathrm{pca} + J_\mathrm{NCX} - J_\mathrm{up} + J_\mathrm{leak} - J_\mathrm{diff\_myo\_ss}\right.\nonumber\\
&&+ \left. D_{Ca}\nabla^2Ca_{\mathrm{myo}}\right)
\label{eq:10}
  \end{eqnarray}
}
%
%\noindent

公式中,$J_{\mathrm{rel}}$是从RyRs到dyad单元间的钙离子流通量,$J_{\mathrm{lca}}$是从L-型钙离子通道到dyad间的钙离子流通量,$\tau_{\mathrm{efflux}}$是dyad空间和膜下空间之间的扩散系数,$J_{\mathrm{NCX}}$是钠离子和钙离子交换在膜下空间产生的电流通量,$J_{\mathrm{diff\_myo\_ss}}$是肌质与膜下空间之间的扩撒流通量,$J_{\mathrm{diff\_ds\_ss}}$是dyad与膜下空间之间的扩撒流通量。$J_{\mathrm{diff\_NSR\_JSR}}$是NSR和JSR间的扩撒流通量,SR泵可以将钙离子从NSR抽到SR中,产生的流通量用$J_{\mathrm{leak}}$来表示。$J_{\mathrm{pca}}$是流过胞质膜的钙离子流通量,膜下空间会被SR和SL瞬间缓冲,瞬间缓冲系数是通过$\overline{B_{\mathrm{ss}}}$来表示的。$\overline{B_{\mathrm{JSR}}}$是CSQN到JSR间的瞬间缓冲系数,$\overline{B_{\mathrm{myo}}}$是从CMDN和TRPN到胞质膜的瞬间缓冲系数。更加详细的关于各个参数和变量的信息可以参考文献\upcite{gaur2011multiscale}。

\subsection{数值策略}
本课题采用随机的方法对L-型钙离子通道和RyRs进行随机模拟,详细的介绍将在\ref{cellImp}中介绍。显式时间积分法被用来求解所有的微分方程(\ref{tissuevoltage})-(\ref{eq:10})。涉及的扩撒项采用中心有限差分的方法进行离散化。而对于单域方程(\ref{tissuevoltage}),使用的是算子分裂方法\upcite{qu1999advanced}。这意味着扩撒项与离子产生的电流项$I_\mathrm{ion}$是分开处理的,后者是通过求解Ord模型中的所有常微分方程得到所有的离子电流,并将所有的离子电流累加起来而得到。

所有的计算流程是发生在一个时间循环内,循环中是每个时间步需要完成的工作,具体包括对所有细胞内的模拟,然后是根据公式\ref{tissuevoltage}对细胞间电压扩撒过程进行计算。每个细胞内的计算包括对所有的dyads进行计算,然后对dyad间的扩撒过程进行计算。具体过程如表\ref{overview}中的伪代码所示,对其中的计算进行高效实现变得非常重要。

\begin{table}
\caption{3D组织模拟计算的伪代码实现}
\label{overview}
\begin{lstlisting}[language=C++, basicstyle=\ttfamily\footnotesize]
Global initialization
  for (int t = 0; t < time steps; t++) {	
    for (int k = 1; k <= cells; k++) {
      Cell computation
      for (int j = 1; j <= dyads; j++) {
        L-type probability calculation
        L-type opening	
        RyR probability calculation
        RyR opening
        Ca concentration computation }
      Dyad diffusion }
    Cell difusion }
\end{lstlisting}
\end{table}

\section{基于多核CPU系统的心脏组织3D模拟的实现}

 \subsection{多级并行}

在介绍大规模心脏组织模拟的实现之前,先介绍实现所面向的目标平台以及相应的编程模型。本课题面向的是多核CPU集群系统,集群中节点间通过以太网或者光纤高速互联,节点间通信可以通过MPI实现,在单节点内也可以创建多个MPI进程进行通信。本课题采用了MPI进行编程,每个MPI进程控制一个计算节点,节点间通过MPI进程进行通信,这种编程方式可移植性好,方便程序员显式控制通信。对于单节点内部来说,每个节点都是由多核CPU构成的,为了充分利用多核CPU的计算资源,本课题采用OpenMP的编程方法,为每一个CPU核分配相应的任务。对于其中的CPU核来说,由于CPU支持向量化执行,因此,还可以进一步利用硬件的并行计算资源。

基于现有的目标平台以及编程模型,本课题采用了一种多级并行策略对大规模心脏组织的模拟进行加速。首先是将3D心脏组织网格划分成很多子网格,每一个网格中的所有细胞的模拟分配给一个计算节点完成,而对于网格中的所有细胞的模拟,由于各个细胞的模拟是相互独立的过程,因此可以采用OpenMP对细胞的模拟并行执行。MPI通信只发生在组织层,因为在计算扩散项\ref{tissuevoltage}时,当前细胞的电压与相邻的细胞的电压有关,这就会涉及到MPI通信了。然而主要的计算还是发生在细胞内, \label{cellImp}将重点介绍单个细胞内的计算的实现。
 
 \subsection{单个细胞内的数值计算实现}
 \label{cellImp}
 在细胞内的所有计算中,最耗时的计算部分是关于钙离子处理的那部分。因此,本节重点关注这部分计算,暂且将这部分计算命名为{\tt computeCalciumInDyad},表\ref{cellfunction}中展示的是该部分计算的伪代码。
 
 \begin{table}
\caption{The function that implements calcium handling per cell.}
\label{cellfunction}
\begin{lstlisting}[language=C++, basicstyle=\ttfamily\footnotesize]
void computeCalciumInDyad()
    {
        generateRandData();
        for(int i=0;i<Ndyads;i++) {
            computateLocalLtypeCurrent();
            computeLocalSRCaRelease();
            computeLocalCaConcentration();
        }
        computeCaConcentrationDiffusion();
    }
\end{lstlisting}
\end{table}

 函数{\tt  computeCalciumInDyad}的计算复杂性直接与dyads的数目成正比,还可以看到函数{\tt  computeCalciumInDyad}中调用了五个函数,其中函数{\tt generateRandData}负责产生大量的随机数,每个时间迭代步中都需要重新产生新的随机数,这些随机数在细胞内L-通道的开和关的模拟中被使用到。函数{\tt  computeCaConcentrationDiffusion}是用来计算细胞内dyad间钙离子扩散中的浓度变化。另外三个函数将在一个{\tt for}循环中调用,特别的,函数{\tt computeLocalSRCaRelease}中的主要部分是计算每个dyad内$100$个RyRs状态间的随机转换过程。在接下来的内容中,将展现三个对并行模拟器性能产生影响的编程和数值技术。
 
 \subsubsection{避免重复计算}
在每个dyad的计算中,都需要计算大量的变量,有些变量在不同的dyad内值是不一样的,因此,对于这些变量,每一个dyad发生的计算中都应该包含这些变量的计算。然而,对于有些变量,其值对于所有的dyad都是相同的,由于dyad数量最多可以达到$10000$,,如果在进入对dyad的{\tt for}循环之前,预先将这些在每一次时间迭代步内不会改变的变量的值计算出来,可以大大减少计算的次数。这需要从这些复杂的公式中找出循环常量,根据实验的性能提升可以发现这些努力是值得的。

\subsubsection{细胞内的dyad间扩散的向量化}
细胞内的$10000$ 个dyads形成一个$100\times10\times10$的网格,钙离子由于浓度不同会在这些dyad间扩散。表\ref{diffusion}中的伪代码是这种3D扩散的计算实现过程,代码主要由三个{\tt for}循环构成,其中计算主要集中在最内的循环,最内的循环次数为$100$,因此可以考虑对该循环进一步优化。由于CPU核已经在细胞级采用OpenMP并行了,每个CPU核负责细胞内的计算,如果需要对细胞内的计算进一步并行,只能开发CPU核内的并行计算资源了。本课题采用的是Intel CPU,每个CPU核内具有256位的向量计算单元,所以对于64位双精度浮点计算来说,每个CPU一次可以处理4个双精度的浮点运算。由于dyad内的数据在$x$方向连续存储的,最内循环适合进行向量化。本课题通过添加编译指导语句,借助编译器对特定的循环进行向量化优化。理论上应该可以取得4倍的加速,但由于扩散中的计算是存储受限的计算,实际取得的性能提升将远低于理论上能取得的性能。

 \begin{table}
\caption{Pragma guided vectorization (in the $x$ direciton) of one of three diffusion computations between the dyads in function {\tt computeCaConcentrationDiffusion}.}
\label{diffusion}
\begin{lstlisting}[language=C++, basicstyle=\ttfamily\footnotesize]
for(z=1;z<Nz_diff-1;z++)
        for(y=1;y<Ny_diff-1;y++) {     
           int x,c,n,s,b,t;         
           x=0;
           c=x+y*Nx_diff+z*Nx_diff*Ny_diff;
           n=c-Nx_diff;            s=c+Nx_diff;
           b=c-Nx_diff*Ny_diff;    t=c+Nx_diff*Ny_diff; 
           
           U[c]=u[c]+fx*(2*u[c+1]-2*u[c])+fy*(u[n]+u[s]-2*u[c])
                    +fz*(u[b]+u[t]-2*u[c]);
           #pragma ivdep
           for(x=1;x<Nx_diff-1;x++) {
              ++c; ++n; ++s; ++b; ++t;
              U[c]=u[c]+fx*(u[c-1]+u[c+1]-2*u[c])+fy*(u[n]+u[s]-2*u[c])
                       +fz*(u[b]+u[t]-2*u[c]);              
           }
           U[c]=u[c]+fx*(2*u[c-1]-2*u[c])+fy*(u[n]+u[s]-2*u[c])
                                         +fz*(u[b]+u[t]-2*u[c]);
        }   \end{lstlisting}
\end{table}


\subsubsection{采用二项分布}
\label{binom}
%
前面的章节中已经介绍了每个dyad内包含$100$ RyRs,每个RyR在任何时刻会处于四种状态中的一种状态,在\ref{cellmodel}已经介绍了这四种状态,以及这四种状态间的转变关系,状态间的转变是按照一定的概率随机转变的。现在的目标是计算RyR中处于打开状态(O1)的数目,因为打开状态的RyR能够决定钙离子流量而影响细胞内电压,一种直接的方法是对每个RyR独立地进行模拟。这样每个dyad内需要消耗100个随机数对这100个RyRs的状态转变进行模拟。每个RyR根据当前状态以及相关的转变概率转变到另一个状态中,这种模拟的方法需要消耗大量的计算,一方面是因为需要消耗大量的随机数,随机数在每个时间步中都需要重新产生;另一方面,这种模拟实现中,代码含有大量的{\tt if}测试语句。

因此,如果能将dyad内的100个RyRs看作一个整体,直接计算出处于开放状态的RyR的数目,将有效降低计算量。因此,本课题将原本100次独立的随机试验替换为在一个二项分布中进行八次随机抽样的实验,对图\ref{ryrstates}中的每一次转变进行一次抽样试验。在二项分布中,$n$次试验中成功$k$次的概率为$p$,则$p$可以用公式\ref{eq:cumulative}计算。

\begin{equation}
F(k,n,p)=Pr({X}\leq{k})=\sum_{i=0}^{k}\binom{n}{i}p^i(1-p)^{n-i},
\label{eq:cumulative}
\end{equation}
式中$$\binom{n}{i}=\frac{n!}{i!(n-i)!}$$为二项系数。由于处于各个状态的RyRs的数目是从$0$ 到$100$间变化,可以预先计算好所有的二项系数,这也能显著提高计算性能。

使用服从$[0,1]$的均匀分布的随机数,对二项分布进行采样找出最小的$k$,使得$r \leq F(k,n,p)$。然而标准的二项分布实现对计算需求比较大,本课题采用了一个高效的实现,如表\ref{fig:binom}中代码所示。因为二项分布的系数预先已经计算出来了,只需要在每一轮试验中将$p/(1-p)$与基概率$(1-p)^n$相乘,并与二项系数相乘然后保存结果。

 \begin{table}
\caption{Implementation of Binomial Distribution Method. Function {\tt constant\_p\_binomial} simply finds $k$ by using the precomputed lookup table. In function {\tt binomial},
we first test if the computation can be skipped, using the threshold $p_t$ and the precomputed value $F(0,n,p_t)$. If this is not the case, the distribution function $F(k,n,p)$ is computed iteratively by subtracting from {\tt randValue} using the precomputed binomial coefficients.}
\label{fig:binom}
\begin{lstlisting}[language=C++, basicstyle=\ttfamily\footnotesize]
    int constant_p_binomial(n,randValue) {
        k=0;
        while(randValue>Table[n,k]) 
            k++;
        return k;
    }

    int binomial(n,p,randValue) {
        	if n = 0
		    return 0;
        if p < Threshold AND randValue < Precomp[n]
            return 0;
        k = 0;
        p_current =pow((1-p),n);
        p_step = p/(1-p);
        while(randValue > 0) {
            k++;
            randValue -= Binom[n,k]*p_current;
            p_current *= p_step;
        }
        return k;
    }  
           \end{lstlisting}
\end{table}

为了表示方便,用$x_1$,$x_2$,$x_3$,$x_4$这四个变量代表RyR处在四种状态的数目,并用$x_{ij}$表示RyR从状态$i$到状态$j$间的数目,$x_{ij}$是通过上述描述的方法通过对二项分布采用计算出来的。因此,在下一个时间步,处于各个状态的RyR数目可以用公式\ref{transitNumber}计算。

\begin{equation}
\label{transitNumber}
x_i = x_i - \sum_{j}x_{ij}+\sum_{j}x_{ji}.
\end{equation}

基于此,本课题根据细胞模型的特点增加了两个优化方法,在本课题中提出的细胞模型中,RyR中的状态中,从O1到C1和从C3到C2的转化概率为常数,即$p_c=0.5*dt$。因此,可以仿照预先计算二项系数的思路,也预先计算全部的累积概率函数$F(k,n,p_c)$。在这两种方法中,唯一的代价是需要存储$101*100/2=5050$个双精度浮点值,这增加了一些额外的存储开销。

另外,本课题还利用了细胞模型中另一个特性,就是RyR中大部分时间都是处于C2状态的,并且状态间的转变概率除了上述所说的两个为常数概率,其它转变概率在大部分时间都是接近0的。这意味着通过从二项分布中采样计算出的转变的RyR的数目也接近0。本课题通过设置一个小的概率阈值记为 $p_t$,并且对于所有的$0 \leq n \leq 100$,预先计算出$F(0,n,p_t)$ 。如果$p \leq p_t$,只要检查随机数$r$,看是否满足$r \leq F(0,n,p_t)$,如果满足,则$k=0$,代表没有发生状态转变,就可以不用计算二项分布了。当然,如果转变前处在某个状态的RyR数为0,那么发生状态转变的RyR数也为0,详细的代码参见表\ref{fig:binom}.

采用二项分布对RyR状态转变进行模拟的方法也在文献\upcite{restrepo2008calsequestrin}中被采用了,然而作者并没有去计算实际的二项分布,而是采用正态分布和泊松分布对其进行逼近,不过这会降低模拟的精度。
   
   
\section{实验结果与分析}
\subsection{实验设置}

本课题的测试系统是在Abel\upcite{abel}机器上,这是一个由奥斯陆大学管理维护的一个超级计算机系统。在Abel机器上的每个计算节点由Intel Xeon E5-2670处理器构成,共有16个物理计算核心,每个计算核心的频率为2.6GHz,由FDR Infiniband互联(56 Gbps)。本课题最多使用了128个计算节点(即2048个CPU核)进行数值模拟试验。本课题采用Intel的\textit{icc} 15.1.0编译器,Intel MPI 5.0.2库进行通信。

在所有的试验中,无论是组织层级还是细胞层级都采用大小为$0.05$ ms的固定的时间步。在组织层级模拟中,对于公式\ref{tissuevoltage}中扩散项的离散化,本课题选择了$0.5$ mm的固定的网格分辨率。

\subsection{性能优化实验}
第一个数值计算的试验是为了测试\ref{cellImp}中在函数{\tt compute\_cell}中介绍的那些优化方法的性能提升效果。为了测试这些优化方法的性能,对包含$10000$个dyad的细胞模拟$10000$个时间步,等价于模拟一次持续时间为$500$ ms的心脏跳动。模拟期间,在$t=50$时刻给细胞一次刺激。图\ref{fig:optimizations}给出了各个不同的优化方法的性能提升。通过避免重复计算可以显著提高各个函数的性能,其中函数{\tt computateLocalLtypeCurrent}的性能提升$37.5\%$,函数{\tt computeLocalSRCaRelease}的性能提升$24.4\%$,函数{\tt computeCaConcentrationDiffusion}的性能提升$12.4\%$。向量化能进一步加速扩散部分的计算,加速约$25\%$。最后是采用二项分布的方法对RyR通道的模拟的影响,从试验结果看,二项分布方法对性能影响最大,能过加速SRCaRelease约$70\%$,在随机数产生方面可以提升$79.9\%$,对于后者的影响完全是因为二项分布的方法可以有效减少对随机数的使用。最后,在所有的方法都使用后,总的时间减少了约$50.7\%$。

\begin{figure}[htb]
\center
\includegraphics[width=\textwidth]{figs/optimization.pdf}
\caption{Performance improvement of the individual functions in {\tt computeCalciumInDyad} due to the different optimization techniques. The three optimizations are applied cumulatively. Thus, the values for the Binominal method reflect the sum of all improvements.}
\label{fig:optimizations}
\end{figure} 

\subsection{扩展性实验}
本章对弱扩展和强扩展都进行了试验,试验中模拟时长为$1000$ ms,对于两个扩展试验,分别都做了两个模拟试验,一个是细胞内的dyad数量设置为$100$个,另一个是细胞内dyad数量设置为$10000$。

在弱扩展试验中,本课题使用的计算节点的数目从$1$到$128$。当使用$100$个dyad时,为每个计算节点分配的心脏组织大小为$64\times64\times64$,$128$个计算节点上完成大小为$512\times256\times256$ 的心脏组织的模拟。对于$10000$个dyad来说,为每个计算节点分配的心脏组织大小为$16\times16\times16$,$128$个节点分配心脏组织大小为$128\times64\times64$。扩展性试验中,每秒钟能够完成的细胞计算量作为扩展性的衡量标准,一个细胞计算量是指一个时间步计算一个细胞。图\ref{scaling1}给出了弱扩展试验的结果,其中纵坐标代表细胞计算量,横坐标是使用的计算节点数。从结果看出,试验表现出很好的弱扩展性。

对于强扩展试验中细胞内dyad设置为$100$个时,心脏组织大小固定在$256\times256\times256$这个规模,而对于每个细胞内dyad数量为$10000$,心脏组织规模为$32\times32\times32$,由于存储的需求,强扩展试验中最少需要$8$ 个计算节点。与弱扩展试验中类似,仍然采用细胞计算量这个指标,图\ref{scaling1}也展示了强扩展试验的结果。强扩展基本上取得了与弱扩展一样的性能。性能上细微的差别可以通过通信量来解释,强扩展试验中通信量会比弱扩展试验大些,而在现有的实现中,通信暂且未做优化。

由于心脏细胞模拟是计算密集型的应用,因此,无论是在弱扩展,还是强扩展试验中,模拟试验都表现了很好的扩展性。通过每秒细胞计算量这个指标,可以对任何规模的心脏组织在给定计算资源的条件下需要模拟的时间进行预测。

\begin{figure}[htb]
\center
\includegraphics[width=\textwidth]{figs/scaling.pdf}
\caption{Performance of weak and strong scaling tests of tissue simulations. The $Y$ axis shows performance measured via the number of cell computations (i.e.~time steps for a single cell) performed for each wall-clock second of simulation time used. }
\label{scaling1}
\end{figure} 


\subsection{细胞内离子活动实验}

在图\ref{fig:calciumcell}中,我们展示了人类心脏细胞内正常的心跳过程中钙离子处理变化过程。试验以不同的尺度展示了心脏组织的中心细胞内钙离子处理结果。图\ref{fig:calciumcell}中A部分显示了在两次连续的心脏中活动电势的变化;图\ref{fig:calciumcell}中B部分是位于细胞中心的dyad内处于打开状态的RyR的数目,这些数目是按照\ref{binom}通过二项分布的方法计算出来的。模拟结果表明,在活动电势变化期间,大部分RyR是出于打开状态的,随着这些通道的打开,在dyad内的$Ca_{ds}$浓度迅速上升到$500\mu$M,以及$Ca_{ss}$上升到$0.1$mM,具体变换过程可以从图\ref{fig:calciumcell}中的C和D观察到。对于亚细胞层级,对细胞内的Ca ($Ca_{i}$)进行模拟的线扫描影像表明,细胞内所有dyad内的钙离子释放是同步发生的(参考图\ref{fig:calciumcell}的F)。整个细胞内相关的$Ca_i$为图\ref{fig:calciumcell}中G所示,以及H未整个细胞的JSR。总之,这些结果描述了正常心脏跳动过程中的不同尺度的钙离子处理过程,包括dyad级、亚细胞级以及细胞级。

\begin{figure}[htb]
\includegraphics[width=\textwidth]{figs/calcium.pdf}
%\includegraphics[scale=0.8]{figs/calcium.pdf}
\caption{Calcium handling in a cell in a human tissue during normal excitation. (A) Action Potential. (B) Number of open RyRs ($N_oRyR$) in the center dyad of the cell. (C) Dyadic space calcium ($Ca_{ds}$) (D) Submembrane space calcium ($Ca_{ss}$) (F) Simulated linescan image of intracellular Ca ($Ca_i$ (G) Whole-cell $Ca_i$ and (H) Whole-cell junctional sarcoplasmic reticulum (JSR). The 3D tissue was plane-stimulated at an edge at a cycle length(CL)=500 ms. Last two steady-state beats are shown.}
\label{fig:calciumcell}
\end{figure}

\subsection{心脏组织内异常活动模拟}

图\ref{arrhythmia}在3D心脏组织中的膜电位变化图,在时刻t = 0 ms,心脏组织的边缘受到一个刺激,经过一段时间的电压传播,在t = 600 ms时,心脏组织的一部分细胞受到新的刺激,在新刺激和之前的电压扩散的双重影响下,最后在心脏组织内部形成一个卷波。从心脏组织中的三个位置A、B、C的细胞进行采样,这三个位置的膜电位变化如图\ref{arrhythmia}中下半部分所示。在整个模拟过程中,卷波非常稳定,没有产生新的小波。这个模拟试验表明了本课题实现的这个组织模拟器能够模拟出在心脏组织的异常功能研究中非常有用的卷波。

\begin{figure}[htbp]
\includegraphics[width=\textwidth]{figs/voltage}
\caption{Simulation of scroll waves in a 3D tissue. The top part shows voltage snapshots at different time points. The bottom section shows membrane potentials at three different locations in the tissue.}
\label{arrhythmia}
\end{figure}

\section{小结}
本章介绍了3D组织模拟的精细模型,采用了细胞内多尺度钙离子处理模型对心脏组织细胞的电活动和钙离子处理过程进行模拟。以往针对3D心脏组织的许多研究都是利用全细胞钙离子处理模型,而多尺度的钙离子处理模型并没有在组织模拟中广泛使用,其中的原因一方面是因为多尺度模型模拟对计算需求巨大;另一方面是当今缺少针对这些计算的有效的数值方法和算法。而本课题提出了一个多尺度模拟的模型,并在大规模的集群系统实现了。在实现过程中,本课题采用了多种优化方法,首先是减少了心脏组织细胞的模型中的很多变量的重复计算,其次是根据硬件平台的并行计算资源对3D心脏组织进行了多级并行,最后对心脏组织模拟的核心模拟过程采用了二项分布的方法。这些优化方法使得模拟的时间减少了一半。对性能提升效果最明显的是二项分布方法的引入,二项分布方法减少了对随机数的使用,可以节省大量用于产生随机数的时间。

弱扩展测试针对不同的组织规模,在不同数目的计算节点上基本保持了相同的计算时间。而强扩展试验中,随着计算节点数目的增加,计算时间基本保持了一个线性下降的关系。这些结果表明了将更大规模的心脏组织模拟映射到更大规模的集群系统中进行模拟成为可能。最后通过试验证明了本课题研究针对心脏组织细胞内的钙离子处理模拟与已经发表的工作\upcite{o2011simulation}的结论是一致的。试验也证明了本课题研究的模型能够模拟3D心脏组织的心率失常等问题。

总结本章工作,本章提出了一个精细的3D心脏组织模拟模型,这个模型能够模拟人类心脏组织细胞内的电活动和钙离子处理过程,通过采用多个优化方法,最后将模型在大规模多核系统高效地实现了。试验中的扩展性结果表明全心脏的模拟在当前最快的超级计算上进行模拟也是可能的。在下一章中,本课题将进一步提高3D组织模拟器的性能,并且将已有的实现移植到更快的硬件加速器系统中。这将逐渐从多个尺度增加研究人员对心脏病理的机制的理解。





\chapter{心脏组织模拟在天河2异构系统上的并行优化}
\label{icpads}

\section{引言}
在\ref{ICA3PP}中,已经介绍了心脏组织模拟在研究心律失常等各种心脏疾病具有重要的作用,心脏组织模拟的模型越详细,对计算的需求越大,在\ref{ICA3PP}中也已经介绍了本课题在多核CPU的集群系统上做的一些工作,采用了MPI+OpenMP的并行编程方法,对一定规模的心脏组织进行了模拟。但如果想对更大规模的心脏组织模拟则需要计算能力更强的处理器或者加速器构成的超大规模集群系统。文献中\upcite{GPUcell}采用了GPU在细胞层级对模拟进行加速,利用加速器对模拟进行加速不失为一种好的选择。

基于现有的计算资源,本课题将心脏组织模拟移植到超级异构集群系统即天河2号超级计算机\upcite{tianhe}上。天河2号超级计算机系统在TOP500排名中位居第二,它的每个计算节点都是由Intel CPU与Intel PHI协处理器构成的异构系统。

众核加速器的每个芯片集成了大量的简单小核,非常适合作为高性能计算系统的核心计算部件。Intel的Xeon Phi协处理器和GPU是两种典型的众核加速器。本课题将面向天河2号超级计算机系统实现心脏组织的3D模拟,天河2号超级计算机系统采用多核CPU作为控制核心以及Intel Xeon Phi协处理器作为加速部件。天河2号系统的峰值性能可以达到54.9PFlops,实测性能可以达到33.86PFlops。

天河2号系统具有很强的计算能力,但天河2号作为复杂的异构系统,这给天河2号上的并行编程带来了诸多挑战。因此,对于一个应用能否在天河2号上取得很好的性能,取决于并行实现。对于天河2号系统来说,除了需要高效地充分利用其大量的硬件线程,单个物理线程上的SIMD向量化也需要尽可能地得到发挥。物理线程的并行可以通过OpenMP编程实现,或者通过一个MPI进程内创建多个OpenMP线程控制多个物理线程,SIMD向量化理论上能够将双精度的浮点性能提升8倍,SIMD向量化可以在代码中添加编译指导语句,借助编译器对代码自动向量化,或者采用AVX-512向量指令\upcite{XEONPHI}手动对代码进行向量化。编译器自动向量化的方法只针对代码中比较规则的计算,而对于不规则的代码则需要通过手动向量化实现SIMD。

对于由CPU和Xeon Phi构成的异构系统来说,多核CPU的性能同样不容忽略,目前每个计算节点内CPU核的个数一般在8到18之间,然而每个CPU核使用更高的时钟频率,功能也更加强大,更加灵活,适合做一些控制的任务。对于一些特殊的计算,比如很难实现SIMD或者是存储带宽受限,则多核CPU上的实现性能可以达到协处理器上的实现性能。因此异构节点上CPU核与Xeon Phi协处理器应该协同起来,充分发挥异构系统的性能。然而,从编程的角度来说,异构计算可能带来新的问题,比如,对于SIMD向量化,CPU采用不同的AVX指令集,因为CPU的SIMD指令是针对256位的向量计算。异构计算还可能需要解决优化CPU和Xeon Phi间的数据传输问题以及CPU和Xeon Phi间的负载均衡问题等。

本课题的目标是将心脏组织的模拟移植到天河2号的超级异构计算机系统中,为了有效地使用天河2号的硬件资源,需要将模拟的计算过程进行多层次的并行开发,包括计算节点间的并行、节点内CPU和Phi设备间的并行、Phi设备上众核间的并行以及Phi设备的每个计算核心内SIMD的并行。这无疑会给编程带来巨大的挑战,这正是本课题需要解决的问题。

本章内容安排如下,首先将介绍天河2号超级计算机系统的硬件结构;然后将介绍心脏组织模拟在天河2号系统上的高性能并行实现与优化,主要利用硬件结构的特点,将心脏组织模拟的并行映射到具体的硬件中执行,并对其中的关键计算部分进行优化;最后是对试验结果评测,主要测试了心脏组织模拟在天河2号上单节点内的Phi设备上的性能、单节点的性能以及多节点的性能。试验结果表明,本课题针对天河2号异构集群系统开发出了一个高效的心脏组织模拟器。

\section{目标硬件体系结构}

天河2号超级计算系统包含16000个计算节点,每个计算节点有3个Intel Xeon Phi协处理器和2个Intel Xeon E5-2692v2 Ivy Bridge CPU处理器。因此,天河2号系统共有48000个Xeon Phi以及3200个Intel Xeon CPU。Xeon Phi的模型为31S1P(即Knigths Corner体系结构),每个Xeon Phi有57个核心,频率为1.1GHz,每个CPU有12和核心,频率为2.2GHz。

Intel Xeon Phi协处理器是一个基于众核设计的硬件加速器,与传统的多核CPU不同的是,Phi协处理器是由大量的性能相对较弱的小核构成的,这些小核以环形的结构相连,每个小核支持512位宽的向量处理,具有64KB大小的L1 cache以及512KB大小的L2 cache,所有小核的cache支持一致性协议。Xeon Phi的这种设计使得它不适合处理串行程序,但非常适合处理并行程度高的程序。对于Xeon CPU来说,CPU中的每个小核具有64KB大小的L1 cache,256KB大小的L2 cache,并且Xeon CPU具有被12个小核共享的30MB大小的L3 cache。

Xeon Phi的理论峰值双精度性能可以根据参数理论上计算出来,它的向量位宽为512位,所以每次可以处理8个双进度浮点运算或者16个单精度浮点运算,根据前面介绍的频率和小核的个数,可以计算出一个Xeon Phi的单精度浮点峰值性能可以达到$1001$ GFlop/s。然而,在实际中,一般不能取得这么高的性能,CPU的计算性能也不容忽略,峰值性能可以达到$211$ GFlop/s,实际中应该充分发挥Xeon CPU和Xeon Phi两种计算资源的性能。

对于节点间的通信问题,天河2号系统采用了自己定制的名为TH Express-2的互联系统,这是一个胖树的拓扑结构。理论上说,Xeon Phi上是支持MPI通信的,单节点内Xeon Phi间采用MPI通信对性能不会有太大损失,但对于处于不同节点的Xeon Phi间的通信,采用MPI通信并不是很高效的方法。因此,本课题采用的策略是一个节点由一个MPI进程控制,然后在MPI进程内为每一个Xeon Phi创建一个线程,每个线程负责管理一个Xeon Phi,Xeon CPU和Xeon Phi间的通信采用Intel的SCIF/COI接口\upcite{SCIFCOI}进行通信。由于SCIF/COI接口较底层,编程会相对较复杂,但会取得很好的性能。

Xeon Phi的57个小核中,有一个小核是保留给操作系统使用的,因此,真正能用于计算的小核个数只有56个。每个小核能同时运行4个物理线程,但对于给定的一个线程,它不会在连续的两个时钟周期都运行,意味一个Xeon Phi内只有有效的112个线程能以$1.1$ GHz的时钟周期运行。当运行224个线程时,每个线程是以系统的时钟周期的一半运行的。Xeon Phi协处理器的DDR5设备存储大小为8GB,小核访问设备全局存储的峰值带宽\upcite{xeon_phi_peak}为$320$ GB/s。然而,即使在理想情况下,对于很多应用来说,峰值带宽的一半还是可以达到的\upcite{testdrivingPhi}。


\section{心脏组织模拟的高性能并行实现与优化}

。。。。。。。。

\subsection{心脏组织模拟的tissue-级并行}
\label{tissueParallel}
心脏组织模拟的全部计算任务具有多个层次的并行性,结合目标体系结构的特点,本课题对心脏组织模拟的计算任务进行分解,分解成多个层次。在最顶层分解中,心脏组织细胞以3D网格的方式结合在一起。每一个计算节点被分配到一定数量的细胞进行处理,这些细胞构成一个子域,子域的形状也是一个立方体。处于子域边界的细胞,它们的电压值在进行扩撒计算前需要与相邻的子域边界的细胞的电压值交换。细胞的电压值交换是通过CPU间的MPI通信实现的,采用了标准的halo交换方法。

对于每个子域在节点内的CPU和Phis间的细胞划分,本课题没有对立方体构成的子域进行规则的划分,因为这会导致CPU与Phis间负载的不均衡,而是根据CPU和Phi的计算能力分配相应比例的细胞给CPU和Phi这两种计算设备。这种策略需要对CPU和Phis处理的所有的细胞的电压值进行交换,CPU和Phi间的通信采用Intel的SCIF/COI通信接口实现。虽然这种方法会增加节点内CPU和Phi间的通信,但付出的代价会比因为负载不均问题产生的额外开销小。图\ref{nodeload}是节点内对子域构成的立方体细胞划分的示意图,不同颜色的部分代表分配给不同的计算设备。

 \begin{figure}[bth]
\center
\includegraphics[scale=0.5]{figs/cube}
\caption{Example of the intra-node decomposition of a subdomain. Cells allocated to the Phis are shown in \textcolor{blue}{blue}, \color{teal}{teal}, \textcolor{black}{and} \textcolor{cyan}{cyan}, \textcolor{black}{while the CPU cells are shown in} \textcolor{yellow}{yellow}.}
\label{nodeload}
\end{figure}

另外,因为每个细胞内含有大量的dyad单元,在代码实现中,每个细胞大约需消耗3MB的存储,受限于Phi设备的8GB的DDR存储以及Phi设备同时最多运行224个线程,导致每个Phi设备最多只能处理2500个细胞,这意味着每个线程每个时间步也只能计算少量的细胞,这使得整体性能对负载均衡非常敏感。因此,为每个Phi设备分配的细胞数最好是224的整数倍,这有利于设备内线程的负载均衡。

在前期的试验结果发现天河2号的每个计算节点内,2个CPU的性能与单个Phi实测的性能比约为$1.5 : 1$。根据上述限制,本课题为每一个节点分配的细胞规模为$20\times20\times20$,$8\times224=1792$ 个细胞分配给一个Phi设备,剩下$8000-3\times1792=2624$个细胞分配给主机上的2个CPU进行计算。CPU处理的细胞数大约占单个节点内任务的$32.8\%$,同样的比例同样适用其它大小规模的子域。然而,实际测试中,针对不同的子域,CPU和Phi设备的性能还是存在细微的差异,因此,要做到完美的负载均衡是不太可能的。

 
\subsection{心脏组织模拟的cell-级并行}
正如\ref{tissueParallel}中提到的负载均衡问题,本课题采用了一种直接而静态的分配策略,即事先将细胞固定指定到相应的计算设备中执行,并且细胞内的所有计算自始自终都在同一个小核中完成的。与静态分配策略相对应的是动态分配策略,每个细胞内特别是所有dyad内的计算由所有的计算核心共同完成,这种方法可以更好地缓解负载均衡的问题。但由于部分计算必须串行执行,在计算过程中会导致重要的延迟。

因此,在每个小核中,本课题采用SIMD向量单元并行处理4个(CPU)或者8个(Phi)细胞内钙离子释放单元的计算。向量化对模拟实现的性能具有重要的影响作用,由于模拟过程的复杂性,本课题采用混合并行化的方法,包括由Intel icc编译器完成的自动编译方法和利用Intel 向量化指令实现的手动向量化方法。为了实现混合并行化的方法,对源代码实现中的dyad循环分割成多个小的循环将变得很有必要。因为dyad内部的计算中,既有很规则的计算部分,记为\textit{arithmetic}段,也有部分含有大量控制语句的代码,记为\textit{conditional}段。对于循环中的规则计算,编译器可以很容易对其向量化,然而,对于大量控制语句的代码,编译器很难实现自动化。因此,在对dyad循环内部的计算分割成多个循环之后,规则的循环可以通过编译器自动向量化,而非规则的计算通过手动向量化,这可以大大简化并行优化的工作。

上述说的\textit{arithmetic}段的计算包括决定L-型通道或者RyR状态转变概率、计算钙离子浓度以及计算dyad间的扩散。它们的计算都比较规则,只需要在循环外加上适当的编译指导语句,编译器就能对这些代码自动实现向量化。虽然,dyad间的扩散部分是存储受限并且只包含少量的计算,向量化同样适用,只是提升的性能不是那么显著。\textit{conditional}段包含的从二项分布中采样决定RyR中发生状态转变的数目,由于这部分计算需要{\it while}循环或者等价功能的代码,编译器对这部分代码不能自动实现向量化。

将大循环分割成多个小循环技术的弱点在于降低了数据局域性,对于大循环来说,只需要存储最终的计算结果,因为循环能保证一次性将结果计算出来,而当大循环分割成很多小循环之后,最终的计算结果依赖中间各个阶段的计算结果,需要额外的数组存储这些中间结果,临时中间结果在前半部分被计算出来写到全局存储中,在后半部分计算又从全局存储中读出,这增加了存储带宽的压力。为了避免反复访存,可以考虑将部分循环合并,这就需要手动实现合并之后循环内所有的计算代码。

\subsection{二项分布采样的向量化实现}
对于\textit{conditional}中的代码,可以采用AVX-512指令进行手动向量化,向量化条件语句代码的关键就在于强大的\texttt{mask}指令。\texttt{mask}指令允许SIMD向量单元在执行向量指令时,只对部分元素按照指令执行,针对的元素取决于一个位掩码。AVX-512指令只能在Phi设备上执行,在Sandy Bridge结构的CPU上缺少相应的AVX指令,因此,在针对CPU的代码中,二项分布采样的实现中未进行手动向量化。

AVX-512向量化的实现可以参考图\ref{Vecbinom}。该向量化代码是通过同时计算8个dyad的累积概率分布函数实现采样的,以{\tt N, P}和{\tt RANDVAL}两个向量作为输入,它们包含处在某状态的通道数、转化概率以及针每个dyad采样的一个随机数。采样的结果{\it K},即改变状态的RyR的个数是通过将{\tt RANDVAL}减去二项分布采样的累积概率,直至结果变为负值,需要经过的循环次数$i$而得到。当{\tt RANDVAL}中的某个分量达到负值时,对应的{\tt MASK}将被设置成$0$,而{\tt K}中对应的分量将保持不变。RyR和L型通道转变的处理采用了相同的二项分布采样的方法,为了可读性的方便,这里只介绍了RyR的采样过程。

\begin{figure}[hbt]
\small\begin{verbatim}
function VectorizedBinomial
Input: Vectors N, P, RANDVAL
Output: Vector K
   Initialize K = 0
   Initialize 1P = Vector_subtract(1,P);
   Initialize PKNK = Vector_power(1P,N);
   Initialize P1P = Vector_divide(P,1P);
   for (int i = 0; i < max(N); i++) {	
       BC = Vector_gather(BC_table,N,K);
       SUB = Vector_multiply(BC,PKNK);
       RANDVAL = Vector_subtract(RANDVAL,SUB);
       PKNK = Vector_multiply(P1P,PKNK);
       MASK = Vector_mask_compare(RANDVAL > 0);
       K = Vector_mask_add(K,1,MASK); }
\end{verbatim}
\caption{The vectorized version of the binomial sampling computation. It can be derived from a scalar implementation via the use of MASK to increase the output value for some vector elements, instead of using a \texttt{while} loop. Vector intrinsics are renamed for clarity.}
\label{Vecbinom}
\end{figure}

为了节省时间,向量\texttt{PKNK}和向量\texttt{P1P}在初始化会预先被计算出来。由于向量{\tt N}中的每个元素最大值不会超过$100$,二项分布系数的值可以在真正模拟之前预先计算出来并存储在{\tt BC\_table}。这种方法对于CPU这种只适合标量处理的处理器是有利的,但对于每个小核访存带宽受限,计算能力充足的Phi可以消除这项技术的优势。

向量化实现的方法中,最主要的问题就是计算一直持续到最后一次采样结束。这意味着向量化取得的加速效果取决于处在同一个向量中的8个dyad的采样结果差异,如果所有采样结果相差不大,则可以获得近8倍的加速效果,但实际中,这种情况很少出现。另外,测试掩码向量\texttt{MASK}的值是否为0,即所有的采样已经结束,这种条件检验尤其在Phi设备上代价是比较大的。为此,采用了一种折中的方案,即每隔8次循环进行一次条件判断,来决定是否完成采样过程。

这意味着向量{\tt K}中的最大元素越小,采样时间就越短,这种情况也是经常发生的,因为大都RyR状态间的转化概率是很小的,因此,采样结果一般都是比较小的。另外,向量{\tt N}中的某个元素为0,则向量{\tt K}中对应的元素也一般为0,如果是这种情形,则可以不需要计算就得出结果,随机数向量{\tt RANDVAL}可以在后面的采样使用,这可以节约随机数使用的数目。并且注意到,在图\ref{ryrstates}中的状态转化中,从O1状态到C1状态以及从C3状态到C2状态间的转化概率是常数,这意味着累积分布函数是可以预先计算出来的,这可以大大降低采样过程的计算量。

二项分布的采样过程确实不是特别适合向量化处理,采用大量的柏努利试验方法决定RyR通道处于打开状态的方法更有利于向量化的实现,这是由Phi上的\texttt{compare}指令的特点决定的。然而,这种直接的方法需要消耗大量的随机数,随机数生成在Phi设备上并不是特别的高效。因此,本课题并未采用柏努利试验的方法。

\subsection{心脏组织模拟中随机数生成}
为了产生二项分布采样所需的随机数,本课题使用了Intel MKL库中提供的函数\texttt{vdRngUniform},并在每一次时间步迭代之前调用该函数。随机数产生过程在Phi设备上的性能相对CPU上的性能差很多,单核上的性能几乎相差$40$倍。另外,与CPU不同的是,在Phi设备上随机数生成在多核上的扩展性也非常差,这使得随机数生成成为Phi设备上计算的瓶颈之一。

由于\texttt{vdRngUniform}是预先定义好的库函数,本课题并不能对其实现做进一步的优化。然而,正如上面所提到的,可以在使用随机数的时候做些优化,比如,对于通道数为0的各个初始状态转变的计算是可以跳过的,因此,就不需要消耗随机数了,减少了总的随机数的产生,从而减少了随机数产生的时间。为了实现这个方法,使用一个变量记录每次时间步内所使用的随机数的数目,使用了多少随机数,就在下一个时间步中产生多少随机数覆盖掉当前时间步中使用过的随机数,当前时间步未被使用的随机数将在下一个时间步中继续使用。对于每个dyad来说,RyR通道各个状态的数目计算最多需要消耗8个随机数,L型通道状态的计算最多消耗2个随机数,因此每个dyad最多需要消耗10个随机数,采用上述减少随机数的方法之后,平均能减少一半的随机数的使用,意味着随机数产生的时间可以降低一半。

\subsection{代码优化}
尽管x86 CPU和Xeon Phi在体系结构上具有很强的相似性,但在这两种设备上移植程序而都保持很高的性能并不是件简单的任务。在Xeon Phi设备上较低的串行程序性能,较长的向量单元以及很高的计算性能这些特点,与存储带宽上的性能的对比,使得在这种设备上编程时需要重点解决一些挑战。

Xeon Phi加速器计算性能的发挥非常依赖于512位向量单元得到充分利用,完全的向量化实现能使Xeon Phi的双精度性能直接提升8倍,单精度性能提升16倍。对Xeon CPU来说则是,向量化能使双精度性能提升4倍,单精度性能提升8倍。在大部分情况下,可以在代码中插入编译指导语句使得编译器对相应代码实现自动向量化,而对于复杂的代码,则需考虑通过程序员采用向量化的指令手动向量化实现。

对于Xeon CPU来说,AVX向量化不能对所有的双精度的计算类型都提高4倍,因为只有两个双精度除法的硬件单元可以同时执行,正如文献\upcite{notallflopsequal}中所讨论的。这是因为在CPU处理器中,除法几乎比乘法慢20倍,向量化的加速比最后就取决于在除法运算中获得的加速比了。而Xeon Phi中的每个小核能并行处理8个双精度除法运算,不过除法的开销也同样代价较大,因此无论是针对Xeon CPU还是Xeon Phi设备,都应该尽量减少对除法的使用,比如可以用乘法代替除法,不过,这都需要手动实现,因为编译器无法自动实现乘法操作对除法的替换工作。

对于细胞内的dyad间的扩散计算,编译器也可以实现该部分计算的自动向量化,并且能够取得一定性能的提升。一种有效的模板计算的实现是避免使用条件分支语句,因为待执行的指令不是数据相关的。然而,由于这些操作都是访存受限的操作,期望获得的性能提升一般低于计算密集型指令取得的性能提升。

\section{实验结果与分析}
\subsection{实验设置}
所有的试验都是在天河2号超级计算机上执行的,采用Intel icc 14.0.2编译器进行编译。节点间通信采用MPICH 3.1.1底层通信库,在使用MPI编程时,每个计算节点上分配一个MPI进程,MPI进程是由节点的CPU主机控制的。每个节点的MPI进程创建3个线程用来控制3个Phi设备,每个线程负责启动Phi设备计算以及控制CPU主机与Phi设备间的数据传输。Phi设备的启动以及CPU和Phi设备间的数据传输都是通过底层的SCIF/COI接口实现的。SCIF/COI编程相对比较复杂,但这能让程序员完全掌控CPU和Phi设备间的通信过程以及Phi设备的计算过程。采用SCIF/COI编程的方法也算是卸载模式的一种,虽然不是通过编译指导语句指定和控制相关在Phi设备上执行的代码。虽然通过编译指导语句的方式也能控制Phi设备的执行,但这种方式没有SCIF/COI编程方式控制起来高效。

对于共享存储的所有小核,无论是对于Xeon CPU还是Xeon Phi,本课题采用经典的OpenMP编程,分别创建24个和224个线程进行计算。在天河2号系统中,对于Xeon CPU来说,超线程功能是关闭的,本课题在Sandy Bridge结构的16核CPU上采用超线程取得了$23\%$的性能提升,因此对于天河2号上Ivy Bridge CPU来说,超线程技术理论上应该可以带来同样的性能提升。

由于处于不同socket的CPU共享存储,因此将双CPU看作单个设备来处理。通过\textit{scatter}方式创建分配线程。天河2号上的Phi设备的时钟频率大都是$1.1$~GHz,但也存在部分只有$0.8$~GHz的设备。由于权限的问题,本课题最多只能用到400个Phi设备。

在所有的试验中,无论是组织级还是细胞级模拟的时间步大小都为$0.05$~ms。对于心脏组织级的模拟,采用分辨率为$0.5$~mm的空间网格对扩散项公式进行离散化。在具体的心脏模拟试验中,模拟一个心跳,即10000时间步($500$ ms),对于大规模扩展试验中,只模拟$200$~ms。所有的试验中,对心脏细胞的外界刺激都发生在时刻$t=50$~ms。为了使得不同的模拟试验中的运行时间具有可比性,试验中采用每秒钟细胞计算数(CC/s)。

\subsubsection{心脏组织模拟单设备性能}
本节介绍了系列试验来研究心脏组织模拟器的性能。第一个实验测试了Xeon CPU和Xeon Phi两种处理器的单核性能,图\ref{fig:optimizations}是试验结果,从结果中可以看出,混合向量化成功地减少了\textit{arithmetic}段的计算时间。在Xeon Phi设备上,钙离子浓度的计算部分被加速了约8倍,这是SIMD方法能加速到的最大速度。L型通道的概率计算部分加速了约6倍,虽然这部分代码是纯粹的计算,但只包含少量的计算,并且涉及一些数据传输,这可能会导致一些延迟。细胞内的扩散部分的计算加速了约5倍,之所以不能达到理想的8倍加速,也是因为该部分计算是存储受限型的计算。最后就是关于RyR概率的计算加速了约4倍,主要的原因是计算中含有大量的指数函数,指数函数的开销是非常大的,而且向量化效果不是特别理想。

对于\textit{conditional}段的计算,由于手动向量化的影响,L型通道转化的计算部分取得了6倍的加速。对于RyR通道转化的计算,手动向量化只使计算加速了约$20\%$,使得这部分计算称为细胞内计算占用时间最长的部分。另外,随机数产生的优化对Xeon Phi影响更加显著。显著的性能提升证明了针对Xeon Phi代码优化的重要性。需要注意的是\textit{conditional}段的代码因为缺少必要的向量指令,在Xeon CPU上并没有进行向量化。

\begin{figure}[htb]
%\begin{center}
\includegraphics[width=\linewidth]{figs/perf1crop.pdf}
%\end{center}
\caption{Performance improvement of the individual functions in the dyad computations due vectorization. For the Xeon Phi, this also includes the conservation of random numbers.
Note the change in scale due to the low single-core performance of the Xeon Phi. Performance is reported per time step, based on a measurement over 10,000 time steps.}
\label{fig:optimizations}
\end{figure}

第二个实验是检验单个Xeon Phi设备上的扩展性,包括弱扩展和强扩展。由于存储的限制,弱扩展实验中让每个线程处理8个细胞,则224个线程最多处理$8\times224$个细胞,基本达到单个Xeon Phi设备的存储上限。在强扩展试验中,单个设备的处理的细胞数也为$8\times224$,考虑到执行时间长短的因素,强扩展试验至少使用的线程数为$16$。图\ref{fig:scaling}是强扩展和弱扩展试验中的使用不同线程数下模拟10000个时间步的执行时间,如果将执行时间转化为每秒钟细胞计算数这个指标,这个指标在强扩展和弱扩展中基本相同,所以强扩展和弱扩展的性能差别不大。从试验结果还可以发现,当使用到56个物理线程时,可以观察到非常好的扩展性,能达到$86\%$的效率。而在使用线程数超过56时,扩展性有所下降,这是由于小核的存储带宽竞争变激烈导致的。当使用到112个线程时,效率下降到$65\%$了。当线程数目超过了物理核数时,除了存储带宽方面的竞争,如果两个线程共享一个物理核,则每个线程的cache大小将减少,这也将影响性能的发挥。当设备中创建224个线程时,两个线程共享时钟周期,虽然没有使用更多的硬件资源,但可以观察到明显的性能提升。从线程使用的角度看,虽然效率下降了,但相对于资源使用的角度看,其实效率是相当高的。这些现象和数据表明,模拟性能很大程度上受限于延迟而不是吞吐量,因此超线程的使用能缓解这个问题。

\begin{figure}[htb]
%\begin{center}
\includegraphics[width=\linewidth]{figs/weakstrongscalePHI.pdf}
%\end{center}
\caption{Weak and strong scaling on a single Xeon Phi, shown as running time over 10,000 time steps for $8$ cells per thread in weak and 1,792 cells in strong scaling.}
\label{fig:scaling}
\end{figure}

在Ivy Bridge CPU上做了同样的强扩展和弱扩展试验,唯一不同的是模拟的规模,弱扩展试验中,每个小核负责8个细胞的计算,强扩展中需要完成$8\times24$个细胞的计算。在Ivy Bridge CPU上取得与Xeon Phi上相似的扩展性,除了Ivy Bridge CPU不支持超线程试验,因此被使用的物理核的数目与创建的线程数时钟保持一致。详细的试验结果可以参考图\ref{fig:scaling2}。扩展性因为存储带宽达到饱和而受限,即当每个socket中使用的物理核数超过两个时,性能没有与使用的物理小核数保持线性增长。

\begin{figure}[htb]
%\begin{center}
\includegraphics[width=\linewidth]{figs/weakstrongscaleCPU.pdf}
%\end{center}
\caption{Weak and strong scaling on the CPU, shown as running time over 10,000 time steps for $8$ cells per thread in weak and 192 cells in strong scaling. Threads are distributed evenly over both sockets.}
\label{fig:scaling2}
\end{figure}

\subsection{心脏组织模拟的单节点性能}
第三个实验测试了单个节点在心脏组织模拟计算中的性能。节点的每个Xeon Phi设备被分配$8\times224$细胞,Xeon CPU上分配$2624$个细胞,共$20\times20\times20$个细胞,这些细胞构成一个三维的网格。表\ref{tbl:perf1}是试验结果,注意到,每次都是分配细胞给节点上的两个CPU主机,因此测试的是2个CPU的性能。显然,单节点内的Phi的性能是相对稳定的。然而,由于技术原因,节点内的各个Phi的时钟周期会有所差异,因此,在某些节点上负载均衡问题会比预期的差些,导致CPUs处于等待状态。

\begin{table}
\caption{Single and aggregate performance of the devices within a node. CC/s is the number of cell computations per second.}
\label{tbl:perf1}
\begin{center}
\begin{tabular}{c|ccc}
Device & Cells & Time (in sec.) & CC/s \\
\hline
2 CPUs	&	2,624	&	1,706	&	6,152	\\	
1 PHI	&	1,792	&	1,859	&	3,856	\\	
2 PHIs	&	3,584	&	1,860	&	7,708	\\	
3 PHIs	&	5,376	&	1,861	&	11,555	\\	
2 CPUs+3 PHIs	&	8,000	&	1,879	&	17,030	\\	
\end{tabular}
\end{center}
\end{table}

图\ref{loadbalance}表示的是负载均衡的情况。虽然节点内的通信占用了不是特别重要的时间,但不完美的负载均衡可以导致CPU等待约$10\%$的计算时间。可以看出异构计算对负载均衡问题是非常敏感的,实际上,如果CPU展示的性能比预期的还差,则因为Phi而导致等待的时间将变得更大。由于单个节点每次处理的细胞都是立方体的限制,一个完美的负载均衡是不可能达到的。

\begin{figure}
 \includegraphics[width=\linewidth]{figs/loadbalance.pdf}
  \caption{Load balancing within the node. The communication overhead is insignificant, but the imperfect load balancing is noticeable. (Percentages which are not show consist of compute time for all devices.)}
  \label{loadbalance}
\end{figure}

\subsection{心脏组织模拟在多节点上的扩展性试验}

Based on the performance we achieved on one node, we tested our code on multiple nodes to assess the influence of MPI communication and load balance on scalability. Figure \ref{weak_scale} shows the weak scalability from 1 to 512 nodes. We achieve near-perfect scalabilty due to low MPI communication. In addition, load balancing between the nodes 
has only a minor effect on performance in this experiment.   

Each run uses the standard subdomains of $20\times20\times20$ cells arranged in a cuboid manner. For the 400 node run which simulates a flat slab of tissue containing 
$400\times400\times20$ cells, the communication overhead was slightly lower at $1.76\%$ due to the 2D nature of the communication there. Note that due to machine limitations, communication time for the 512 node run was measured using only CPU computation and performance we extrapolated from the previous measurements.

\begin{figure}
 \includegraphics[width=\linewidth]{figs/weakscaling.pdf}
  \caption{Weak scaling of the overall simulation. The communication overhead remains below $3\%$, allowing for near perfect scaling.}
  \label{weak_scale}
\end{figure}

Based on the rate of increase, we estimate that when using all 16,000 nodes, communication will still amount to less than $6\%$ of the total running time, making this code well suited for extreme scale simulations.

\subsection{心脏组织模拟试验}

\section{小结}
