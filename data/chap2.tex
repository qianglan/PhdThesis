%*********************第二章******************
\chapter{使用``目标候选''提高跟踪器的尺度和宽高比适应力}
\label{chapbmvc}

\section{引言}
介绍尺度和宽高比适应力是什么,重要性。

介绍什么是目标候选。本章方法是通用的,这里仅使用最适合跟踪的EdgeBoxes(本章主要介绍跟踪器部分,更多目标候选介绍在下一章)。

分点介绍本章讲什么。

\section{相关研究}
\subsection{跟踪器的尺度和宽高比适应力}
介绍具有尺度和宽高比适应力的典型跟踪器
\subsection{基于相关滤波的跟踪器}
介绍典型相关滤波跟踪器

\section{使用核化相关滤波器进行视觉跟踪}
翻译并简化kcf论文的核心部分

\section{图像特征整合和鲁棒更新}
bmvc论文

\section{将``目标候选''嵌入跟踪器中}
bmvc论文

\section{实验结果与分析}
\subsection{实验设置}
\subsubsection{参数设置}
\subsubsection{测试集和对照组构成}
介绍OTB50,介绍自己提取的子集
\subsubsection{评价标准}
介绍CLE和IoU,介绍精确度图和成功率图
\subsection{尺度和宽高比适应力评测}
\subsection{整体性能评测}

\section{小结}
