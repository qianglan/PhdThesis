%*********************第一章******************
\chapter{绪论}

\section{研究背景和意义}

\subsection{视觉跟踪}
什么是视觉跟踪,用途,价值,困难挑战

\subsection{高性能视觉跟踪}
两方面的高性能

\subsubsection{高性能视觉跟踪算法}
解释含义,及其价值、挑战

\subsubsection{跟踪算法的高性能实现}
解释含义,及其价值、挑战

\subsubsection{异构计算平台及其编程模型}
什么是异构平台(当前火热),介绍GPU和CPU(MIC)体系结构,介绍几个常用编程模型



\section{研究现状}

\subsection{经典视觉跟踪算法的相关研究}
给出经典跟踪器结构,介绍几大模块的发展

\subsection{新兴视觉跟踪算法的相关研究}
TLD,神经网络的应用,``目标候选''方法的应用

\subsection{高性能跟踪算法实现的相关研究}

\subsubsection{异构平台下跟踪算法的高性能实现}
介绍几个跟踪器的高性能实现(没有使用OpenCL实现的),指出缺少可移植性(不能在各种平台下运行)

\subsubsection{异构平台下的高性能编程模型}
介绍几个高性能跟踪器可用的新兴编程模型,指出移植性问题。说明OpenCL好处,指出OpenCL的性能移植性问题,列出几个解决性能移植性的相关工作


\section{主要研究内容和创新点}
1. 一个将目标候选融入跟踪器的方法

2. 分析目标候选对跟踪精度的影响,从而对通用的目标候选生成方法进行改进

3. 基于OpenCL的高性能TLD跟踪器实现

4. 解决OpenCL程序从GPU到CPU的性能移植性问题

\section{论文结构}