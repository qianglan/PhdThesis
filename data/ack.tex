
%%% Local Variables:
%%% mode: latex
%%% TeX-master: "../main"
%%% End:

\begin{ack}
在我的博士论文完成之际,谨向在我攻读博士学位的过程中曾经指导
过我的老师,关怀过我的领导,关心过我的朋友,和所有帮助过我的人们致以崇
高的敬意和深深的谢意!

首先要感谢我的导师张春元教授。
从2009年第一次踏入课题组,如今已近8年。
在这8年中,在张老师的指导和关怀下,我渐渐从一个一无所知、唯唯诺诺的新手学徒,
成长为自立自强、独立思考的工科学者。
作为导师,张老师具有远见卓识,对计算机学科各个领域都有着丰富的经验和准确的把握。
常常在我迷茫之时为我指明方向,在我困惑之时为我厘清思路,在我怯懦之时助我树立信心,
在我冲刺之时铸我坚强后盾。
作为长辈,张老师平易近人,勤俭务实,睿智执着,言传身教。
在生活上给予我细心关怀,教给我大量为人处世的道理,帮助我战胜无数纷繁复杂的困难。
作为朋友,张老师温和谦逊,风趣幽默。
在课题组,在办公室,在小路边,在聚会上,张老师常与我促膝长谈,
或是闲聊逗乐,或是推心置腹。
张老师的工作作风、处世态度和奋斗精神无一不感染着我,成为我一生的财富。

衷心感谢课题组的文梅研究员。
8年间,文老师亲力亲为,事无巨细,不断细心指导我,鞭策我,
帮助我在学术之路上披荆斩棘,奋发图强。
文梅老师为我提供了细致的研究指导,顶尖的研究条件,充足的研究资料,宽松的研究氛围。
从我进入课题组到接近博士毕业,可以说每一行代码,每一个文字,
都有文老师的悉心指导;
每一点成就,每一篇论文,都有文老师倾注的心血;
每一次懈怠,都有文老师的有力鞭策;
每一次萌生退意,都有文老师的真诚鼓励。

感谢课题组的罗磊师兄,他带领我走入了计算机视觉的研究,为我指明了研究方向和道路,
帮助我解决了无数难题,指导我取得了如今的成果。
罗师兄踏实勤奋,分析问题深入透彻,指导耐心尽责,为我提供了巨大帮助。
还有荀长庆师兄,他带领我开始了系统软件的研究,
指导我学习理论知识,与我一同编写代码,为我的动手能力和学习能力打下了坚实的基础。
荀师兄思维敏捷,做事雷厉风行,工程能力极强,执行力、集中力突出,是我学习的榜样。
还有伍楠师兄,从我硕士开始,就为我提供了无数宝贵的意见和建议,帮助我少走了很多弯路。
伍师兄带领我认识了产业界的雄心壮志、奋发图强、激烈竞争,帮助我认清了自身的差距,为我提供了不断前进的动力和勇气。

感谢美国加州大学圣地亚哥分校的Scott Baden教授,尽管我在美国一年期间,主要从事自己的研究,
总是自说自话。但他从不介意,还为我推荐课程和老师,提供极佳的学习研究环境,帮助我解决生活上的困难。
还有挪威Simula实验室的Xing Cai教授,他治学严谨,认真细心,亲切和蔼。
他帮助我逐字逐句修改的论文,是我真正开始科技写作的第一步,为我奠定了写作能力的基础。

感谢课题组的所有成员。同一级的蓝强、乔寓然、薛云刚是我的挚友,是奋斗的战友,也是生活的伙伴。
感谢任巨、杨乾明、苏华友、全巍、管茂林、柴俊、贾文涛、陈照云、曹壮、沈俊忠、时洋、吕倩茹、王彦鹏、王泽龙、杨浩铎、黄友、
王自伟、刘甚灵、何义、李京旭、陈东、
肖涛、王丽璇、邓皓文、方浩、辛思达、范方圆、施自龙、杨静、董辛楠,他们伴我一路走来,
给予了我宝贵的支持和帮助。

感谢六院七队的首长:欧阳登轶、寻兵斌、孙靖、王艳丰、孙友佳。他们的管理和关怀极大影响了我的生活态度和作风养成,
提高了我的综合素质。
感谢室友欧洋,作为挚友为我提供了无数的帮助。
感谢同队好友侯富、孙懿淳、姜加红、汝承森、翦杰、贺华鑫、张理勇、林帅、邵则铭、石巍、张文喆、孙振、乔林波、
孙浩、叶永凯等等,和他们朝夕相处十分愉快。

感谢计算机学院和计算机系为我提供了良好的学习、科研环境。学院的每一位老师都十分优秀,是我学习的榜样。
感谢学院机关,他们的勤奋工作是我能够安心学习的保证,是我取得成果的基石。

最后我要衷心感谢我的父母和亲人,常年在外求学,亏欠他们太多太多。
他们是我一切进步的精神支柱,是我不懈奋斗的动力源泉,是我人生中的坚强后盾。

再次感谢所有帮助和关心过我的人!




\end{ack}
