
%%% Local Variables:
%%% mode: latex
%%% TeX-master: "../main"
%%% End:

\begin{ack}
回首博士五年的学习生活,感慨万千,更多的还是想感谢身边的人,因为他们从各个方面为我提供了很多的帮助,让我成长了很多,博士期间取得的成果离不开他们。

首先我要感谢王意洁教授,王老师对学术研究要求严格,对学术研究充满热情,工作勤勤恳恳,王老师这些精神影响着我,让我对学术研究充满敬畏之心,让我丝毫不敢放松对自己的要求。感谢王老师在我开题时给我提出的宝贵意见,以及感谢王老师对我的课题研究的帮助。在生活上,王老师非常随和,还记得王老师在教师节邀请我们一起聚餐,和我们学生谈天说地,讲她自身的经历,俨然像个朋友。临近毕业之时,王老师非常关心我联系工作情况,衷心感谢王老师的关心与帮助。

还要感谢张春元教授,从2010年加入张老师领导的MASA课题组,到现在有八年多了,依然还记得进课题组前,张老师介绍课题组流体系结构等研究方向,那个时候一无所知,还没有做研究的概念,更不知道如何做研究,感谢张老师,让我在课题组学到了很多,成长了很多。喜欢张老师上的高级体系结构课程,张老师幽默、创新的教学方法让我印象深刻。每次跟张老师汇报课题进展,张老师都能从一个独特的视角为我指出一些值得研究的问题,并询问我还有没有困难需要他帮忙解决的,记得有一次我说需要使用天河巨型机,张老师很快就帮我联系解决,感谢张老师在我课题研究中提供的帮助。在平时的聚会聊天中,张老师经常以一种幽默风趣的方式传达着他的人生态度,张老师那种平易近人、乐观向上以及保持对新鲜事物好奇的精神深深影响着我。

特别需要感谢文梅研究员,进课题组的八年来,感谢文老师全程指导我的研究工作,感谢她让我从一个懵懂无知的本科生蜕变成现在即将毕业的一名博士。文老师对学术执着,热爱研究工作,带领我们踏踏实实攻下一个一个的课题项目,让我在项目实践中提高了自身的工程能力和课题方向调研能力。文老师对我们要求严格,为我们指明大概的研究方向,也给予我们足够的自由让我们自己探索找到具体的研究点。在我博士课题研究过程中,文老师陆续指定多名硕士和我一起开展课题研究,我们一起讨论,一起解决问题,感谢文老师给予我这样一个团队环境,提升了我的团队意识和带领团队的能力。博士期间前往挪威Simula实验室进行项目合作一年,感谢张老师和文老师为我提供这样一个交流的机会,拓展了我的视野,提高了我的国际学术交流能力。

非常感谢Simula实验室的Xing Cai教授、Johannes、Namit、Marcus、Vamsidhar、张伟、王帅以及在挪威认识的中国朋友雷大江、姜晓君、林晓琳、宋晖、刘艳君、盼盼、张帆、蒋丽。在挪威交流期间,Xing Cai教授对我进行指导,对我在高性能计算领域的研究帮助非常大,他不放过研究中的任何细节,对于我实验中出现的问题,他要求我一定要找出原因,给出合理的解释,感谢Xing Cai教授,让我在科研工作中养成了认真细心的习惯;感谢Johannes和Namit,他们是我在挪威的合作者,我们一起完成了三篇文章的发表;
%感谢Marcus和Vamsidhar,他们是我在Simula认识的国际友人,感谢他们组织的烧烤、徒步旅行等活动,让我融入他们的生活,了解了他们的文化;感谢张伟和王帅,他们在我刚到实验室在生活上给了我很大的帮助;感谢雷大哥,在我刚到挪威时,是他给我耐心的介绍挪威的情况,让我很快适应了挪威的生活,有半年的时间我们一起坐地铁上班一起做饭,感谢他教会了我做很多菜,也感谢他,让我认识了其他小伙伴们;感谢姜晓君,感谢她第一个周末带我去市中心逛街买东西,感谢她给我分享她做的很多好吃的,现在还记得她做的绿豆糕和鸡翅;感谢林晓琳,她就像姐姐一样关心照顾我,感谢2015年春节她包的饺子,让我感受到与家一样的温暖;感谢宋晖和刘艳君两口子,他们多次邀请我们去他家聚餐,给我们做好吃的,认识他们让我知道了未来我的目标生活,晖哥是我的榜样;感谢盼盼,感谢她邀请大家去她的新家吃火锅为我送行,最后走的早上还是吃的她送的饺子,也感谢她的小麻将,为生活增添了不少乐趣;感谢张帆同学,帆哥虽然话不多,但都是心里话,而且总是用实际行动表达爱意,感谢他多次下了班特地来我们实验室陪我打台球和乒乓球;最后还要感谢蒋丽同学,在Simula实验室,雷大哥走了之后就只有她陪伴了,感谢蒋丽最后走的时候送我上火车,帮我拉行李箱,这种待遇以前从没碰到过。
非常感谢在挪威认识的这些朋友,因为他们的存在,我从没感觉到孤独。

感谢博士期间所在学员队的队领导,包括杨红运政委、孙友佳队长、王艳丰政委、欧阳登轶政委、寻彬彬队长、孙靖副政委。王政委本科阶段就当过我们队长,对我们非常关心,生活上就像大哥一样对我们;感谢欧阳政委、寻队长以及孙副政委,他们基本上陪伴了我们整个博士阶段,对我们管理上松弛有度,注重对我们各方面能力的培养,在他们的带领下,博士生队变得更加有序、更加公平公正。

感谢课题组的所有成员,感谢伍楠、任巨、杨乾明、管茂林、荀长庆、罗磊、刘甚灵、何义、李京旭、付剑、吴伟、苏华友、全巍、贾文涛、彭龙、吴东波等各位师兄,感谢一起进入课题组的乔寓然、黄达飞以及薛云刚三位同学和战友,大家风格迥异聚在一起工作了八年多,我们互相帮助一路走来,建立了深厚的战斗友谊,感谢陈东、施自龙、王自伟、陈照云、沈俊忠、曹壮、肖涛、辛思达、范方圆、时洋、方皓、王泽龙、邓皓文、王彦鹏、杨浩铎、黄友、王得光等各位师弟,感谢吕倩茹、王俪璇 、杨静和董辛楠等师妹。工作的大部分时间都是和他们在一起,从他们身上学到了很多,他们也为我提供了很大的帮助。

%感谢课题组的所有成员,工作的大部分时间都是和他们在一起,从他们身上学到了很多。伍楠、任巨、杨乾明、管茂林、荀长庆、罗磊、刘甚灵、何义、李京旭、付剑、吴伟、苏华友、全巍、贾文涛、彭龙、吴东波他们是我的师兄,刚进入课题组每次开会,在他们的讨论中了解课题组的研究项目,了解学术前沿。伍楠师兄无论是学术还是工程能力都让人敬佩,记得八年前伍师兄一直在讨论设计面向万亿次计算的片上多核系统,很多指标设计都是很超前的;任师兄负责课题组的日常事务,像老大哥一样对我们倍加关心;杨师兄在体系结构硬件方向持之以恒,练就了非常深厚的内功,协助张老师和文老师承担了很多大型项目,和我们朝夕相处,给予我们直接的指导;荀师兄指导过我本课毕设,硕士课题方向也是跟着荀师兄做的,荀师兄思维敏捷,每次跟荀师兄讨论,都有一种慢几拍的感觉(荀师兄口头禅:你怎么听不懂呢?),我都要对自己很无语了,更不用说荀师兄了(我对你无语了),感觉自己把荀师兄的耐心都用完了,衷心感谢荀师兄没有放弃我。管师兄和贾师兄分别做的编译和多核系统的可靠性方向,他们各自为课题组撑起了一个方向,除了张老师和文老师他们真的是一个人在战斗,每次开会,从他们向文老师的汇报中能学到很多概念、专业术语,感谢他们让我可以在外面吹吹牛;罗磊师兄在自己研究方向上研究扎实,还经常关注各种学术报告,了解各个方向动向,在课题组开会时给我们介绍一些相关的工作并给我们提很多建议;刘甚灵、何义、李京旭以及付剑几位师兄虽然毕业多年了,感谢他们给我介绍了很多工作上的经验,谢谢他们的关心;感谢吴伟师兄,记得大四刚进入课题组时,是吴伟师兄给我们做的培训,介绍Stream C/Kernel C编程语言,为我后来异构编程的研究打下坚实的基础;感谢苏华友师兄,苏师兄是计算机学院GPU编程达人,stencil性能优化专家,硕士博士期间碰到很多相关的问题都从苏师兄那里得到解答,苏师兄不仅动手能力前,写论文能力也是一流,感谢他在论文写作上给予的指导;感谢全巍、彭龙和吴东波师兄,感谢我们一起踢球活动的日子;感谢一起进入课题组的乔寓然、黄达飞以及薛云刚三位同学和战友,大家风格迥异聚在一起工作了八年多,我们互相帮助一路走来,建立了深厚的战斗友谊,乔寓然总能把一个问题给人讲明白,很有教授的风范,我们都称他为乔教授,乔教授为了达到幽默的效果说话总是喜欢稍微夸大地说,乔教授在课题研究上能抓住重点,能准确把握住方向,让我深受启发,无论在实验室还是宿舍,我们一起吹牛,聊人生哲学,乔教授某些独特的观点让我耳目一新;达飞同学不仅能力强,还特别刻苦,达飞基本上精通计算机所有方向,达飞很有耐心,经常请教的问题,他都能帮我解答,生活碰到像买电子产品、找餐馆都会找达飞给我推荐,达飞就是那么靠谱、遇到问题首先想到的人,我很崇拜他,可惜他最后找到女朋友了;刚哥学业家庭两不误,总能保持积极的心态面对一切,非常值得我学习;感谢师弟陈东,当时一起负责ZeadBoard开发版的相关实验工作;感谢师弟施自龙帮忙财物报账以及课题组活动的组织;也感谢师弟王自伟帮忙处理课题组一些日常事务,实验室的打印机没墨了都是他搞定的;感谢师弟陈照云,作为课题组日常事务负责人,为我们提供了一个非常好的环境,每次保密检查,设备清查都少不了照云在忙活着,为大家节省了不少事;感谢师弟沈俊忠,大家都叫他呵呵神,呵呵神是我在实验室的陪伴者,篮球场上的球友,每次在实验室与食堂路上,他总能一针见血的评价路上见到的妹子,虽然他有女朋友,但我感觉气质上和我没什么区别;感谢曹壮,他具有丰富的工作经验,平时给我们介绍了很多社会公司的情况,增加我对社会的了解;感谢师弟肖涛,在我去挪威的一年,把我在课题组负责的工程任务以及财务上的事移交给涛哥,辛苦涛哥了,从涛哥身上,我总能看到自己的影子,只是我没有涛哥帅;感谢师弟辛思达和范方圆,他们也协助我处理了很多财务上的事务;感谢师弟时洋、方皓以及王泽龙,我和他们一起开展课题的研究,一起讨论和解决问题,感谢他们对我博士课题的帮助;感谢师弟邓皓文,在深度学习方面,向皓文问过很多问题,对我深度学习入门给予了很大的帮助;感谢师弟王彦鹏、杨浩铎、黄友以及王得光,虽然研究的课题不特别相关,但有时他们碰到问题和我交流之后也让我对很多问题理解更加深入;感谢师妹吕倩茹,师妹是懂生活的人,经常会给我们点好吃的,师妹养的盆摘给实验室增添了不少新鲜空气,平时也跟我提了很多关于课题组活动的建议;感谢师妹王俪璇,虽然课题研究上没有交集,但一起在实验室相处多年,被她乐观的心态、每天大大咧咧的笑声所感染;感谢师妹杨静和董辛楠,她们硕士阶段从事高性能计算方面的研究,为我后来博士课题的研究留下了宝贵的资料和经验。

感谢博士期间多年的同学,感谢同班同学林帅、李翔、刘洋徐瑞、范小康、任双印、邵则铭、石巍、张文喆,感谢二班同学欧洋、侯富、姜加红、汝承森、翦杰、贺华鑫、孙懿淳,感谢他们经常与我们一起组织活动,感谢好朋友毛健彪、乔林波、梁正发、叶永凯、潘璟琨,感谢经常一起打球的同学彭超、陈凯、孙振、单磊、赖超、韩维、樊琐,感谢一起练剑道的同学汪淼,感谢年级骨干蓝龙、张龙、唐川、宝金桢、余越、朱孟斌。

感谢亲戚朋友。感谢我的父母,感谢他们的养育之恩,感谢他们对我的包容与支持。感谢我的表妹林叶青,表弟钟圣达,感谢所有的亲戚朋友,每次回家都少不了他们对我的关心问候。


\end{ack}
