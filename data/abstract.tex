\begin{cabstract}
当前世界正在迎接一波新的人工智能热潮。
人工智能的应用越来越广泛,几乎进入了人们生产生活的所有领域。
计算机视觉对于人工智能,犹如眼睛对于人类一样,获取着绝大部分外界信息,起着决定性的作用。
而视觉跟踪致力于获取运动中物体当前的位置和状态,是大量计算机视觉高层应用的重要基石,
具有基础性的理论意义和无可替代的应用价值。
虽然经过了近三十年的不断研究,
但随着视觉数据的数量和质量不断提高、人们对于智能化的需求不断增加,视觉跟踪仍然面临着众多挑战。
首先,视觉跟踪算法的准确性、鲁棒性和适应力还普遍不足,难以应对纷繁的干扰因素和复杂的物体变化;
其次,为了提高精度和鲁棒性,跟踪算法正日趋复杂,跟踪效率渐渐成为瓶颈;
最后,由于异构计算平台的出现,进行高性能的跟踪算法实现较为困难,且难以向不同的计算设备/平台进行移植。

为了应对上述挑战,
本文以``高性能视觉跟踪关键技术''作为研究课题,希望从两个方面来实现高性能的视觉跟
踪\pozhehao 视觉跟踪的高性能算法和跟踪算法的高性能实现。
本文的主要工作和创新点有:
\begin{compactitem}
\item[1.] \textbf{提出了将物体检测领域常用的``目标候选''方法应用于视觉跟踪的通用方法。}
\end{compactitem}

\begin{compactitem}
\item[2.] \textbf{揭示了``目标候选''在视觉跟踪中的作用规律,并对``目标候选''生成器进行了针对视觉跟踪的优化。}
\end{compactitem}

\begin{compactitem}
\item[3.] \textbf{在异构计算平台上,基于并行编程模型OpenCL,对TLD这一完整视觉跟踪应用进行了高性能并行实现。}
\end{compactitem}

\begin{compactitem}
\item[4.] \textbf{提出了一种新的代码转换方法,以提升GPU特定的OpenCL Kernel程序在CPU上的性能移植性。}
\end{compactitem}



\end{cabstract}
\ckeywords{视觉跟踪; 相关滤波; 高性能实现; 并行编程模型}

\begin{eabstract}


\end{eabstract}
\ekeywords{Visual Tracking; Correlation Filtering; High-Performance Implementation; Parallel Programming Model}

