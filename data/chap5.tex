%*********************第五章******************
\chapter{GPU特定OpenCL Kernel程序的性能移植性提升}
\section{引言}
当前异构平台上的两大类设备(CPU(MIC)、GPU),异构平台上实现高性能跟踪器可用的编程模型,用OpenCL的好处,但是OpenCL编程模型更适合GPU(不考虑数据重组和传输),有性能移植性问题;本章的解决方案,并分点介绍本章讲什么

\section{相关工作}
翻译fitee论文的related work

\section{数组访问的线性描述式}
用线性方程组+约束 描述数组访问

\section{基于分析的Kernel折叠}
以下均翻译fitee论文
\subsection{消除冗余的局部存储数组}
\subsection{依赖性分析和同步语句消除}

\section{适应体系结构的后继优化}
如果前面过长,以下翻译europar论文
\subsection{向量化}
\subsection{局域性重开发}

\section{运行时调度}

\section{性能评测}

\section{小结}