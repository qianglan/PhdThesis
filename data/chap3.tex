%*********************第三章******************
\chapter{3D卷积神经网络在异构平台上的高性能实现}
\label{3DWinograd}
\section{引言}
卷积神经网络在图像分类、目标跟踪等很多2D输入的处理任务已经得到了成功的应用。卷积神经网络能够很好地提取特征,所以卷积网络常用于图像分类,比如Alexnet\ref{}、VGG\ref{} 、googlenet\ref{}、resinet\ref{}等卷积神经网络被用来对2D图像进行分类,达到了很好的分类效果。图\ref{}为经典的Alexnet网络,其主要包含大量的卷积层,其中的计算也主要集中在卷积层。

正因为2D卷积网络得到了广泛的应用,所以研究人员开始转向3D卷积神经网络的研究,在\ref{}已经为我们展现了一个名为ObjectNet3D用于3D物体识别的数据库,通过这个数据库的训练,可以达到识别3D的各种姿势的目的,类似于这样的3D数据库还有ShapeNet\ref{}。针对这些3D数据库的3D卷积网络也开始陆续被设计出来,比如Voxnet\ref{}就是一个用于解决3D物体识别的3D卷积神经网络,文章\ref{}提出用3D卷积神经网络解决人类动作识别的方法,此外, 3D卷积神经网络可以处理视频分类的应用\ref{}。

但对于3D卷积神经网络的应用来说,如果仍然采用2D卷积神经网络的方法来处理,就会存在计算量大,存储消耗大的问题。有些方法只能解决其中的一个问题,比如基于FFT变换的方法在某些情况下可以有效降低计算量,但是以消耗存储为代价的。有一种卷积计算的快速算法,称为WMFA(Winograd Minimal Filtering Algorithm),这种算法目前已经成功运用在2D卷积神经网络中,能够有效降低卷积中的计算量,并且不会增加额外的存储空间。因此,将2D WMFA应用到3D卷积神经网络中是非常值得研究的课题。本章内容主要是介绍3D WMFA算法在3D卷积网络上的应用,需要解决的问题包括,由2D WMFA算法的形式推倒出3D WMFA的形式;从理论上分析3D WMFA算法的复杂性,证明该算法在计算和存储开销上的优势;面向GPU异构平台,将3D WMFA算法高效地映射到GPU上。



\section{相关研究}


\section{快速3D卷级算法}


\subsection{3D卷级神经网络定义}

\subsection{3D Winograd 算法}

\subsection{3D Winograd 算法的复杂性分析}


\section{3D Winograd 算法的实现与优化}

\subsection{3D Winograd 算法的实现}

\subsection{3D Winograd 算法的优化}


\section{实验评测与分析}
\subsection{实验设置}


\subsection{3D Winograd 算法各优化方法的性能}


\subsection{3D Winograd 算法各kernel执行时间分布}


\subsection{3D Winograd 算法与其它卷级算法性能比较}


\section{小结}

