%*********************第三章******************
\chapter{跟踪器中``目标候选''的作用分析和优化}
\label{chapijcv}

\section{引言}
在``目标候选''(原英语术语为``Detection Proposal'',这里采用意译)方法出现之前,
物体检测领域通常使用``滑动窗口(Sliding Window)''的方式来检测物体\upcite{slidingdetect1, slidingdetect2}。
滑动窗口指的是无视输入图像的实际内容,直接用一个窗口扫描整幅图像(在输入图像中不断滑动),
同时不断利用分类器判断窗口中的图像块是否包含目标物体。
由于物体可能是任意大小的,上述过程需要使用不同尺度和宽高比的窗口重复多次。
据统计,在常用的物体检测数据集\upcite{voc2007, coco}上,
一次单尺度的滑动窗口就会产生$10^4\sim 10^5$个候选窗口。
若引入不同的尺度和宽高比,窗口数目将高达$10^6\sim 10^7$\upcite{dpsurvey, dpsurvey2}。
因为整个物体检测过程所需时间等于一个窗口的分类时间乘以窗口数目,
巨大的候选窗口数量严重地限制了用于检测的分类器的复杂度,
使得物体分类领域的先进方法难以应用于物体检测任务。

为了减少滑动窗口所带来的巨大计算压力,``目标候选''方法应运而生。
所有目标候选方法均基于同一个假设,即任何包含物体的图像区域都具有一种共同的视觉特性,
该特性使得这些区域能够和背景区域区别开来。
因此,一定存在某种方法,能够以一幅图像为输入,然后输出一系列很可能包含着物体的候选区域。
这样的方法即是目标候选方法。
如果目标候选方法的输出能够足够准确地覆盖绝大部分物体,
且输出的候选区域数目远小于滑动窗口产生的(这在物体检测领域被称作``高召回率''),
那么物体检测过程将获得巨大的速度提升。
此外,使用更加复杂而准确的分类器也成为了可能,从而检测精度也将大幅提高。
经过多年的发展,目标候选方法已经具备了上述作用,在物体检测领域占有重要地位。
无论在经典的PASCAL数据集\upcite{voc2007}上,还是在庞大的ImageNet数据集\upcite{imagenet}上,
当前排名靠前的物体检测器均使用了目标候选方法\upcite{rcnn, detector1, detector2, detector3}。
实际的应用证明,除了上述作用以外,目标候选方法还能减少``错误肯定(False Positive)''对分类器的干扰,
从而进一步提升物体检测的准确性。

上一章中,通过将目标候选生成器EdgeBoxes不加修改地嵌入KCF跟踪器中,
获得了引人注目的尺度和宽高比适应力提升。
其中的整体方法是通用的,可用于将任何跟踪器和任何目标候选生成器进行结合。
该方法的本质在于,将目标候选生成器作为跟踪器运动模型的补充。
在理想情况下,即所有物体都被目标候选准确覆盖时,这种补充作用将会非常有效:
对于密集的运动模型,目标候选的灵活性将弥补运动模型固化的采样模式;
而对于随机的运动模型,目标候选对物体的感知力将弥补随机采样可能错过跟踪目标这一最大缺陷。
但是,上述的理想情况是难以满足的,目标候选生成器必然存在误差甚至错误,
而目前还尚未有任何文献讨论过目标候选的质量对于跟踪精度的影响。
此外,上一章仅对EdgeBoxes进行了参数调优,并没有对本质算法进行面向跟踪任务的优化,
因此KCFDP的性能还有着一定的提升空间。
本章将针对上述两个问题进行深入的研究。
一方面,通过将多个目标候选生成器逐一地与上一章的跟踪器进行结合,揭示了目标候选的质量和跟踪精度间有着密切的关系。
另一方面,通过为EdgeBoxes加入``背景抑制''这一优化步骤,以极低的时间开销提升了EdgeBoxes对于跟踪任务的适应性,
从而提高了整体跟踪性能。

本章的内容安排如下:
第2节介绍与本章相关的研究工作,主要是典型的目标候选生成器及它们在视觉跟踪中的应用;
第3节介绍目标候选生成器EdgeBoxes,并对其进行面向跟踪任务的优化;
第4节介绍如何将其它5个目标候选生成器和跟踪器相结合;
第5节进行实验,将分析目标候选质量对于跟踪精度的影响,并评测优化后的EdgeBoxes带来的性能提升;
第6节进行本章小结。



\section{相关研究}
\subsection{目标候选生成器及其在跟踪中的应用}
介绍典型目标候选生成器。

介绍使用了目标候选的典型跟踪器。
\subsection{图像分割在跟踪中的应用}
介绍图像分割和目标候选的关系。

介绍使用了图像分割的典型跟踪器。


\section{跟踪器中目标候选生成器的优化}
介绍目的(优化一个目标候选生成器作为``跟踪候选''生成器)
\subsection{目标候选生成器EdgeBoxes}
翻译并简化EdgeBoxes论文的核心部分。
\subsection{使用背景抑制优化EdgeBoxes}
ijcv论文


\section{跟踪器中目标候选生成器的适配}
ijcv论文。介绍目的(为了分析目标候选的作用),介绍5个目标候选生成器,介绍如何修改它们以嵌入跟踪器中




\section{实验评测与分析}
\subsection{实验设置}
主要是参数设置

\subsection{跟踪器中目标候选的作用分析}
ijcv论文,对比不同目标候选生成器的准确度和速度

\subsection{跟踪器中目标候选的优化效果评测}
\subsubsection{统计图对比评测}
以下均翻译ijcv论文
\subsubsection{数值化对比评测}
\subsubsection{VOT测试集上的对比}
\subsubsection{参数敏感性分析}

\section{小结}