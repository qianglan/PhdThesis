%*********************第三章******************
\chapter{3D卷积神经网络在异构平台上的高性能实现}
\label{chapijcv}

\section{引言}
在``目标候选''(原英语术语为``Detection Proposal'',这里采用意译)方法出现之前,

本章的内容安排如下:
第2节介绍与本章相关的研究工作,主要是典型的目标候选生成器及它们在视觉跟踪中的应用;
第3节介绍目标候选生成器EdgeBoxes,并对其进行面向跟踪任务的优化;
第4节介绍如何将其它5个目标候选生成器和跟踪器相结合;
第5节进行实验,将分析目标候选质量对于跟踪精度的影响,并评测优化后的EdgeBoxes带来的性能提升;
第6节进行本章小结。



\section{相关研究}


\section{快速3D卷级算法}


\subsection{3D卷级神经网络定义}

\subsection{3D Winograd 算法}

\subsection{3D Winograd 算法的复杂性分析}


\section{3D Winograd 算法的实现与优化}

\subsection{3D Winograd 算法的实现}

\subsection{3D Winograd 算法的优化}


\section{实验评测与分析}
\subsection{实验设置}


\subsection{3D Winograd 算法各优化方法的性能}


\subsection{3D Winograd 算法各kernel执行时间分布}


\subsection{3D Winograd 算法与其它卷级算法性能比较}




\section{小结}
本章承接上一章的内容,对跟踪器中``目标候选''的作用进行了分析,并对上一章的目标候选生成器进行了深入地优化。
通过将现有的目标候选生成器面向跟踪任务进行适配,
多个生成器被合理地嵌入了跟踪器中。
通过对比这些目标候选生成器在跟踪任务中的表现,
本章证明了目标候选的质量和跟踪精度间存在着正相关的关系,
同时也确认了本章对EdgeBoxes的背景抑制优化是十分有效的。
本章还对最终得到的跟踪器进行了更加充分的实验评测,
其结果表明,本章优化后的目标候选生成器,对于不同速度和剧烈程度的尺度和宽高比变化都有着很强的适应力,
且具有极佳的通用性。
在本章和前一章中,目标候选生成器都是作为密集采样运动模型的一个补充。
下一步的工作将考虑把密集采样运动模型和目标候选生成器进行整合,
以同时发挥密集采样的可靠性和目标候选的灵活性。
